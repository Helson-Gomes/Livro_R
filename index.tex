% Options for packages loaded elsewhere
\PassOptionsToPackage{unicode}{hyperref}
\PassOptionsToPackage{hyphens}{url}
\PassOptionsToPackage{dvipsnames,svgnames,x11names}{xcolor}
%
\documentclass[
  letterpaper,
  DIV=11,
  numbers=noendperiod]{scrreprt}

\usepackage{amsmath,amssymb}
\usepackage{iftex}
\ifPDFTeX
  \usepackage[T1]{fontenc}
  \usepackage[utf8]{inputenc}
  \usepackage{textcomp} % provide euro and other symbols
\else % if luatex or xetex
  \usepackage{unicode-math}
  \defaultfontfeatures{Scale=MatchLowercase}
  \defaultfontfeatures[\rmfamily]{Ligatures=TeX,Scale=1}
\fi
\usepackage{lmodern}
\ifPDFTeX\else  
    % xetex/luatex font selection
\fi
% Use upquote if available, for straight quotes in verbatim environments
\IfFileExists{upquote.sty}{\usepackage{upquote}}{}
\IfFileExists{microtype.sty}{% use microtype if available
  \usepackage[]{microtype}
  \UseMicrotypeSet[protrusion]{basicmath} % disable protrusion for tt fonts
}{}
\makeatletter
\@ifundefined{KOMAClassName}{% if non-KOMA class
  \IfFileExists{parskip.sty}{%
    \usepackage{parskip}
  }{% else
    \setlength{\parindent}{0pt}
    \setlength{\parskip}{6pt plus 2pt minus 1pt}}
}{% if KOMA class
  \KOMAoptions{parskip=half}}
\makeatother
\usepackage{xcolor}
\setlength{\emergencystretch}{3em} % prevent overfull lines
\setcounter{secnumdepth}{5}
% Make \paragraph and \subparagraph free-standing
\makeatletter
\ifx\paragraph\undefined\else
  \let\oldparagraph\paragraph
  \renewcommand{\paragraph}{
    \@ifstar
      \xxxParagraphStar
      \xxxParagraphNoStar
  }
  \newcommand{\xxxParagraphStar}[1]{\oldparagraph*{#1}\mbox{}}
  \newcommand{\xxxParagraphNoStar}[1]{\oldparagraph{#1}\mbox{}}
\fi
\ifx\subparagraph\undefined\else
  \let\oldsubparagraph\subparagraph
  \renewcommand{\subparagraph}{
    \@ifstar
      \xxxSubParagraphStar
      \xxxSubParagraphNoStar
  }
  \newcommand{\xxxSubParagraphStar}[1]{\oldsubparagraph*{#1}\mbox{}}
  \newcommand{\xxxSubParagraphNoStar}[1]{\oldsubparagraph{#1}\mbox{}}
\fi
\makeatother

\usepackage{color}
\usepackage{fancyvrb}
\newcommand{\VerbBar}{|}
\newcommand{\VERB}{\Verb[commandchars=\\\{\}]}
\DefineVerbatimEnvironment{Highlighting}{Verbatim}{commandchars=\\\{\}}
% Add ',fontsize=\small' for more characters per line
\usepackage{framed}
\definecolor{shadecolor}{RGB}{241,243,245}
\newenvironment{Shaded}{\begin{snugshade}}{\end{snugshade}}
\newcommand{\AlertTok}[1]{\textcolor[rgb]{0.68,0.00,0.00}{#1}}
\newcommand{\AnnotationTok}[1]{\textcolor[rgb]{0.37,0.37,0.37}{#1}}
\newcommand{\AttributeTok}[1]{\textcolor[rgb]{0.40,0.45,0.13}{#1}}
\newcommand{\BaseNTok}[1]{\textcolor[rgb]{0.68,0.00,0.00}{#1}}
\newcommand{\BuiltInTok}[1]{\textcolor[rgb]{0.00,0.23,0.31}{#1}}
\newcommand{\CharTok}[1]{\textcolor[rgb]{0.13,0.47,0.30}{#1}}
\newcommand{\CommentTok}[1]{\textcolor[rgb]{0.37,0.37,0.37}{#1}}
\newcommand{\CommentVarTok}[1]{\textcolor[rgb]{0.37,0.37,0.37}{\textit{#1}}}
\newcommand{\ConstantTok}[1]{\textcolor[rgb]{0.56,0.35,0.01}{#1}}
\newcommand{\ControlFlowTok}[1]{\textcolor[rgb]{0.00,0.23,0.31}{\textbf{#1}}}
\newcommand{\DataTypeTok}[1]{\textcolor[rgb]{0.68,0.00,0.00}{#1}}
\newcommand{\DecValTok}[1]{\textcolor[rgb]{0.68,0.00,0.00}{#1}}
\newcommand{\DocumentationTok}[1]{\textcolor[rgb]{0.37,0.37,0.37}{\textit{#1}}}
\newcommand{\ErrorTok}[1]{\textcolor[rgb]{0.68,0.00,0.00}{#1}}
\newcommand{\ExtensionTok}[1]{\textcolor[rgb]{0.00,0.23,0.31}{#1}}
\newcommand{\FloatTok}[1]{\textcolor[rgb]{0.68,0.00,0.00}{#1}}
\newcommand{\FunctionTok}[1]{\textcolor[rgb]{0.28,0.35,0.67}{#1}}
\newcommand{\ImportTok}[1]{\textcolor[rgb]{0.00,0.46,0.62}{#1}}
\newcommand{\InformationTok}[1]{\textcolor[rgb]{0.37,0.37,0.37}{#1}}
\newcommand{\KeywordTok}[1]{\textcolor[rgb]{0.00,0.23,0.31}{\textbf{#1}}}
\newcommand{\NormalTok}[1]{\textcolor[rgb]{0.00,0.23,0.31}{#1}}
\newcommand{\OperatorTok}[1]{\textcolor[rgb]{0.37,0.37,0.37}{#1}}
\newcommand{\OtherTok}[1]{\textcolor[rgb]{0.00,0.23,0.31}{#1}}
\newcommand{\PreprocessorTok}[1]{\textcolor[rgb]{0.68,0.00,0.00}{#1}}
\newcommand{\RegionMarkerTok}[1]{\textcolor[rgb]{0.00,0.23,0.31}{#1}}
\newcommand{\SpecialCharTok}[1]{\textcolor[rgb]{0.37,0.37,0.37}{#1}}
\newcommand{\SpecialStringTok}[1]{\textcolor[rgb]{0.13,0.47,0.30}{#1}}
\newcommand{\StringTok}[1]{\textcolor[rgb]{0.13,0.47,0.30}{#1}}
\newcommand{\VariableTok}[1]{\textcolor[rgb]{0.07,0.07,0.07}{#1}}
\newcommand{\VerbatimStringTok}[1]{\textcolor[rgb]{0.13,0.47,0.30}{#1}}
\newcommand{\WarningTok}[1]{\textcolor[rgb]{0.37,0.37,0.37}{\textit{#1}}}

\providecommand{\tightlist}{%
  \setlength{\itemsep}{0pt}\setlength{\parskip}{0pt}}\usepackage{longtable,booktabs,array}
\usepackage{calc} % for calculating minipage widths
% Correct order of tables after \paragraph or \subparagraph
\usepackage{etoolbox}
\makeatletter
\patchcmd\longtable{\par}{\if@noskipsec\mbox{}\fi\par}{}{}
\makeatother
% Allow footnotes in longtable head/foot
\IfFileExists{footnotehyper.sty}{\usepackage{footnotehyper}}{\usepackage{footnote}}
\makesavenoteenv{longtable}
\usepackage{graphicx}
\makeatletter
\def\maxwidth{\ifdim\Gin@nat@width>\linewidth\linewidth\else\Gin@nat@width\fi}
\def\maxheight{\ifdim\Gin@nat@height>\textheight\textheight\else\Gin@nat@height\fi}
\makeatother
% Scale images if necessary, so that they will not overflow the page
% margins by default, and it is still possible to overwrite the defaults
% using explicit options in \includegraphics[width, height, ...]{}
\setkeys{Gin}{width=\maxwidth,height=\maxheight,keepaspectratio}
% Set default figure placement to htbp
\makeatletter
\def\fps@figure{htbp}
\makeatother
% definitions for citeproc citations
\NewDocumentCommand\citeproctext{}{}
\NewDocumentCommand\citeproc{mm}{%
  \begingroup\def\citeproctext{#2}\cite{#1}\endgroup}
\makeatletter
 % allow citations to break across lines
 \let\@cite@ofmt\@firstofone
 % avoid brackets around text for \cite:
 \def\@biblabel#1{}
 \def\@cite#1#2{{#1\if@tempswa , #2\fi}}
\makeatother
\newlength{\cslhangindent}
\setlength{\cslhangindent}{1.5em}
\newlength{\csllabelwidth}
\setlength{\csllabelwidth}{3em}
\newenvironment{CSLReferences}[2] % #1 hanging-indent, #2 entry-spacing
 {\begin{list}{}{%
  \setlength{\itemindent}{0pt}
  \setlength{\leftmargin}{0pt}
  \setlength{\parsep}{0pt}
  % turn on hanging indent if param 1 is 1
  \ifodd #1
   \setlength{\leftmargin}{\cslhangindent}
   \setlength{\itemindent}{-1\cslhangindent}
  \fi
  % set entry spacing
  \setlength{\itemsep}{#2\baselineskip}}}
 {\end{list}}
\usepackage{calc}
\newcommand{\CSLBlock}[1]{\hfill\break\parbox[t]{\linewidth}{\strut\ignorespaces#1\strut}}
\newcommand{\CSLLeftMargin}[1]{\parbox[t]{\csllabelwidth}{\strut#1\strut}}
\newcommand{\CSLRightInline}[1]{\parbox[t]{\linewidth - \csllabelwidth}{\strut#1\strut}}
\newcommand{\CSLIndent}[1]{\hspace{\cslhangindent}#1}

\KOMAoption{captions}{tableheading}
\makeatletter
\@ifpackageloaded{bookmark}{}{\usepackage{bookmark}}
\makeatother
\makeatletter
\@ifpackageloaded{caption}{}{\usepackage{caption}}
\AtBeginDocument{%
\ifdefined\contentsname
  \renewcommand*\contentsname{Table of contents}
\else
  \newcommand\contentsname{Table of contents}
\fi
\ifdefined\listfigurename
  \renewcommand*\listfigurename{List of Figures}
\else
  \newcommand\listfigurename{List of Figures}
\fi
\ifdefined\listtablename
  \renewcommand*\listtablename{List of Tables}
\else
  \newcommand\listtablename{List of Tables}
\fi
\ifdefined\figurename
  \renewcommand*\figurename{Figure}
\else
  \newcommand\figurename{Figure}
\fi
\ifdefined\tablename
  \renewcommand*\tablename{Table}
\else
  \newcommand\tablename{Table}
\fi
}
\@ifpackageloaded{float}{}{\usepackage{float}}
\floatstyle{ruled}
\@ifundefined{c@chapter}{\newfloat{codelisting}{h}{lop}}{\newfloat{codelisting}{h}{lop}[chapter]}
\floatname{codelisting}{Listing}
\newcommand*\listoflistings{\listof{codelisting}{List of Listings}}
\makeatother
\makeatletter
\makeatother
\makeatletter
\@ifpackageloaded{caption}{}{\usepackage{caption}}
\@ifpackageloaded{subcaption}{}{\usepackage{subcaption}}
\makeatother

\ifLuaTeX
  \usepackage{selnolig}  % disable illegal ligatures
\fi
\usepackage{bookmark}

\IfFileExists{xurl.sty}{\usepackage{xurl}}{} % add URL line breaks if available
\urlstyle{same} % disable monospaced font for URLs
\hypersetup{
  pdftitle={Introdução à Programação e à Ciência de Dados},
  pdfauthor={Prof.~Dr.~Helson Gomes de Souza},
  colorlinks=true,
  linkcolor={blue},
  filecolor={Maroon},
  citecolor={Blue},
  urlcolor={Blue},
  pdfcreator={LaTeX via pandoc}}


\title{Introdução à Programação e à Ciência de Dados}
\author{Prof.~Dr.~Helson Gomes de Souza}
\date{2025-07-07}

\begin{document}
\maketitle

\renewcommand*\contentsname{Table of contents}
{
\hypersetup{linkcolor=}
\setcounter{tocdepth}{2}
\tableofcontents
}

\bookmarksetup{startatroot}

\chapter*{Prefácio}\label{prefuxe1cio}
\addcontentsline{toc}{chapter}{Prefácio}

\markboth{Prefácio}{Prefácio}

Esse é um material construído para auxiliar os discentes do curso de
Ciências Econômicas da Universidade Regional do Cariri no decorrer da
disciplina de \emph{Introdução à programação e à ciência de dados,}
ministrada no terceiro período do curso. O material possui os
instrumentos básicos para o acompanhamento do curso, incluindo
explicações, comandos e exercícios. O material está organizado na ordem
cronológica dos conteúdos repassados na disciplina de tal modo que essa
ordem tem o intúito de conduzir o discente no aprendizado da programação
computacional aplicada à economia em um cronograma planejado de acordo
com as boas práticas do aprendizado da programação, evitando que o
discente comece os estudos por conteúdos inadequados para os iniciantes
nessa área.

O foco do material é a linguagem R, a mesma utilizada como objeto
central da disciplina. Contudo, os métodos de aprendizado também podem
ser expandidos para outras linguagens, desde que as devidas adaptações
de vocabulário sejam consideradas. O material também possui um foco na
análise de dados, desviando-se da programação computacional
\emph{``bruta''} devido a necessidade do curso e dos economistas em
dominar os fundamentos básicos da utilização de dados.

\bookmarksetup{startatroot}

\chapter{Introdução}\label{introduuxe7uxe3o}

\section{A lógica de
programação}\label{a-luxf3gica-de-programauxe7uxe3o}

Imagine que o computador é uma empresa que realiza diferentes tarefas e
que nessa empresa existem funcionários de várias partes do mundo. Cada
funcionário realiza as mesmas tarefas, porém, falando diferentes
idiomas. Assim, se você quer que um funcionário americano realize uma
tarefa, terá que ordenar que ele a faça em inglês; se você quer que um
funcionário italiano realize a mesma tarefa, terá de fazer o pedido para
ele falando italiano, e assim sucessivamente. No computador, os
funcionários são as linguagens de programação e o idioma é o vocabulário
da linguagem (ou a maneira de escrever os comandos em cada linguagem).
Resumidamente e exemplificadamente, você pode executar uma tarefa usando
R ou pode executar essa mesma tarefa usando Python ou qualquer outra
linguagem de programação pois a lógica de comunicação com o computador
será a mesma, porém, os comandos para execuar a mesma tarefa podem ser
diferentes em cada linguagem. Como consequência, o primeiro passo para
dominar as linguagens de programação é desenvolver uma lógica de
programação.

A lógica de programação é um raciocínio sobre como se comunicar com o
computador para realizar tarefas. Seguindo o exemplo da empresa citado
no parágrafo anterior, se você precisa de um relatório impresso, terá de
falar o seguinte \emph{``eu preciso que você imprima esse relatório''}.
Porém, como os funcionários falam diferentes idiomas, você precisa fazer
esse pedido em uma linguagem diferente a depender do funcionário
responsável pela impressão. A maneira como você faz o pedido (\emph{``eu
preciso que você imprima esse relatório'')} não vai mudar independente
do funcionário que receberá a ordem, o que muda é apenas o idioma em que
a ordem será feita. No computador, essa maneira como você faz o pedido é
a lógica de programação, ela será a mesma para qualquer linguagem de
programação, que no caso do exemplo corresponde ao idioma falado.

O primeiro passo para desenvolver uma lógica de programação é entender
as operações fundamentais de uma \textbf{linguagem de programação}. Uma
linguagem de programação é nada mais do que uma maneira de se comunicar
com o computador para ordenar que ele execute tarefas automáticas. As
linguagens são configuradas em sistemas binários (negação e afirmação).
Cada linguagem de programação possui o seu \textbf{vocabulário}, porém,
todas as linguagens partem dos mesmos princípios e das mesmas lógicas
fundamentais. \textbf{Exemplos}: JavaScript, R, Python, Julia, Ruby,
Scala\ldots{}

\section{Como usar a linguagem de
programação}\label{como-usar-a-linguagem-de-programauxe7uxe3o}

O primeiro paso para se comunicar com o computador por meio de uma
linguagem de programação é informar à máquina a sua intenção de usar o
vocabulário da linguagem. Em outras palavras, é preciso ``instalar'' a
linguagem em sua máquina. Na analogia da empresa anteriormente
mencionada, instalar a linguagem de programação seria como fazer um
curso de idiomas e receber um manual de instruções de gramática para
falar com os empregados.

Nesse material, utilizaremos a linguagem R. Seguindo o primeiro passo, a
instalação da linguagem deve ser feita antes de qualquer outro
procedimento. Os métodos de instalação variam de acordo com o sistema
operacional do usuário. Nesse material, vamos supor que o aluno dispõe
de uma máquina com sistema operacional windows. Nesse caso, o usuário
deve baixar os componentes da linguagem
\href{https://cran.r-project.org/bin/windows/base/}{nesse endereço}
Tendo feito isso, o usuário deve executar o arquivo baixado e instalar
normalmente como qualquer instalação convencional no sistema windows.

O segundo passo é adotar um \textbf{ambiente de execução}. Esses
ambientes são softwares também conhecidos como \emph{\textbf{ambiente de
desenvolvimento integrado} (IDE) -} Exemplos: Rstudio, Pycharm, Google
Colaboratory, StataMP-. De forma geral, o IDE é uma plataforma onde você
escreve as ordens que deseja que o computador execute em uma determinada
linguagem. Especificando com outras palavras, o ambiente de execução é
um software usado para escrever as ordens ao computador por meio do
vocabulário da linguagem. Como vamos usar a linguagem R nesse material,
temos que adotar um IDE que execute ordens nessa linguagem. É
recomendado que o usuário utilize o Rstudio apesar de existirem diversas
outras ferramentas com esse mesmo propósito. Para tanto, é preciso
baixar o programa
\href{https://posit.co/download/rstudio-desktop/?cmplz-force-reload=1751898658132}{nesse
endereço} e proceder com com os procedimentos padrões de instalação de
softwares no sistema windows.

É importante ressaltar que uma linguagem de programação não é um
software, por exemplo, é errôneo se referir à linguagem R como
\emph{``software R''}, assim como é equivocado se referir à linguagem
\emph{stata} como \emph{``software stata''}, e assim sucessivamente.
Também é importante destacar que os IDEs não são linguagens de
programação, eles apenas transferem para o computador uma ordem para ser
executado em uma determinada linguagem. Com isso, é errôneo afirmar que
uma dada tarefa foi \emph{``executada pelo software Rstudio''}.

Ao instalar o Rstudio, o usuário irá se deparar com a tela ilustrada na
imagem a seguir. Note que o IDE é formado por quatro painéis, cada um
deles com a sua funcionalidade. O painel 1 é conhecido como \emph{input}
ou painel de entrada. Nesse painel o usuário irá abrir e manipular os
arquivos de entrada como scripts e códigos de programação em R. Outros
arquivos de entrada também são suportados como arquivos \emph{html,
markdown}, dentre outros. Mas essas extensões adicionais não são o foco
do curso. Para gerar um arquivo de entrada onde serão escritos os
comandos da linguagem R, pressione \emph{Ctrl Shift n} ou vá na parte
superior esquerda, na barra de tarefas do IDE em \emph{file, new file, r
script}.

O painel 2 é conhecido como \emph{console}. Nesse painel serão expostos
os resultados das tarefas executadas nos comandos escritos no painel 1.
O usuário também pode escrever e executar os comandos diretamente no
console, com a diferença de que no \emph{r script} os códigos podem ser
salvos para o uso posterior ao contrário do console.

O painel 3 é conhecido como ambiente de trabalho ou \emph{working
environment}. Ele é dividido em quatro abas que podem ser acessadas
separadamente, sendo: o \emph{Environment} onde são expostos todos os
objetos criados, o histórico \emph{(history)} onde os últimos comandos
ficam armazenados numa espécie de breve histórico de comandos, as
conecções \emph{(Connectione)} onde são expostas as ferramentas
conectadas a linguagem R e ao Rstudio para executar tarefas, e a aba de
tutoriais, onde o usuário pode acessar um breve guia sobre como usar o
Rstudio. Versões mais recentes também disponibilizam abas adicionais
sobre controle de versionamento no ambiente de trabalho, mas por
enquanto esse não é o foco do curso.

O painel 4 é conhecido como \emph{output} ou painel de saída. Nele o
usuário pode visualizar os produtos gerados com os comandos e navegar
pelo diretório de trabalho. O painel é dividido em cinco abas, sendo: a
aba de arquivos \emph{(files)}, onde o usuário pode navegar pela pasta
que está usando; a aba \emph{plots}, onde o usuário pode visualizar
gráficos e figuras geradas com os comandos escritos no painel 1; a aba
de bibliotecas \emph{(packages)} onde o usuário pode acessar as
bibliotecas instaladas na linguagem; a aba de ajuda \emph{(Help)}, onde
o usuário pode consultar manuais de instrução sobre diferentes comandos
e bibliotecas; e por fim a aba de visualização \emph{(View)}, onde o
usuário pode visualizar objetos dinâmicos ou estáticos criados com
alguma ferramenta adicional auxiliar à linguagem R como páginas web,
arquivos pdf, etc.

\section{Comandos}\label{comandos}

Os ambientes de execução podem operar em ordens diretas (cliques) ou
comunicação agrupada \emph{(comandos)}. Os ambientes de execução
sugeridos nesta disciplina operam com \emph{linhas de comando}. Nas
linhas de comando, o operador (aluno) escreve uma ordem que o ambiente
de execução entende como um comando para executar uma tarefa. Esses
comandos são escritos individualmente em cada linha, isto é, cada linha
escrita só agrupa um único comando. Ao escrever uma linha de comando no
Rstudio, você poderá executar essse comando posicionando o cursor na
linha (ou selecionando as linhas de interesse) e em seguida pressionando
\emph{``Ctrl Enter''}.

Alguns comandos já estão pré-programados no vocabulário das linguagens.
Porém outros comandos precisam ser elaborados pelo próprio usuário
usando os comandos pré-progamados e as demais ferramentas próprias do
vocabulário da linguagem. Por exemplo, se você quer somar dois números
quaisquer a e b, você pode fazer isso com uma operação de soma que já é
automaticamente reconhecida pela linguagem R. Mas se você quer somar
\(a + b\) apenas se \(a > b\), então você precisa usar a lógica de
programação para programar esse comando.

\section{Bibliotecas}\label{bibliotecas}

O usuário pode ordenar que o computador execute uma tarefa por meio de
comandos. Além disso, o usuário também pode unir vários comandos em uma
função para executar uma tarefa de interesse. Muitas funções já estão
disponíveis na própria linguagem nativa, por exemplo, caso o usuário
queira gerar um gráfico ele pode usar o comando \emph{plot()} que é uma
função que usa vários comandos nativos da linguagem \emph{R} para gerar
uma figura. Assim como a função \emph{plot}, muitas outras funções já
estão disponíveis e prontas para o acesso no ato da instalação da
linguagem. Porém, existem funções que são criadas por terceiros e que
precisam ser instaladas para que possam ser usadas. Por exemplo, em vez
de elaborar um gráfico com a função \emph{plot}, o usuário pode usar a
função \emph{ggplot}. No entanto, essa função só pode ser usada caso a
biblioteca \emph{ggplo2} esteja instalada. O usuário pode verificar se
uma determinada biblioteca está instalada navegando pelo painel inferior
direito, na aba \emph{Packages}, digitando o nome da biblioteca na guia
de busca. Caso a biblioteca esteja instalada, então a busca retornará um
indicativo com o nome da biblioteca, do contrário, o resultado da busca
será vazio.

Para instalar uma biblioteca o usuário deve usar o comando
\emph{install.packages()} posicionando no parêntesis o nome da
biblioteca entre aspas. Por exemplo, para instalar a biblioteca
\emph{ggplot2}, o usuário deve executar o comando
\emph{install.packages(``ggplot2'')}. Feito isso, a biblioteca estará
disponível na lista de bibliotecas instaladas da aba \emph{Packages} do
painel inferior direito. Uma vez instalada, a instalação não precisa ser
refeita, exceto em caso de atualização para uma nova versão.

No entanto, não basta instalar a biblioteca para usufruir de suas
funções. Sempre que o Rstudio for reiniciado ou sempre que uma nova
seção for iniciada no Rstudio é necessário \textbf{\emph{liberar a
biblioteca para o uso}}. Isso é feito por meio do comando
\emph{library()} ou \emph{require()} sempre posicionando entre
parêntesis o nome da biblioteca desejada, dessa vez, sem aspas. Por
exemplo, para liberar a biblioteca \emph{ggplot2} para o uso, proceda
conforme a seguir:

\begin{Shaded}
\begin{Highlighting}[]
\FunctionTok{library}\NormalTok{(ggplot2)}
\end{Highlighting}
\end{Shaded}

\section{Operações fundamentais em
R}\label{operauxe7uxf5es-fundamentais-em-r}

\subsection{Soma}\label{soma}

Para efetuar uma soma, você deve utilizar o operador de adição
\emph{``+''}. Por exemplo, para somar \(1 + 1\) proceda como a seguir:

\begin{Shaded}
\begin{Highlighting}[]
\DecValTok{1}\SpecialCharTok{+}\DecValTok{1}
\end{Highlighting}
\end{Shaded}

\begin{verbatim}
[1] 2
\end{verbatim}

\subsection{Subtração}\label{subtrauxe7uxe3o}

Para efetuar uma subtração, você deve utilizar o traço simples
\emph{``-''} como operador de subtração. Por exemplo, para subtrair
\(1 - 1\) proceda como a seguir:

\begin{Shaded}
\begin{Highlighting}[]
\DecValTok{1{-}1}
\end{Highlighting}
\end{Shaded}

\begin{verbatim}
[1] 0
\end{verbatim}

\subsection{Multiplicação}\label{multiplicauxe7uxe3o}

Para efetuar uma multiplicação, você deve utilizar o asterisco
\emph{``}*'' como operador de multiplicação. Por exemplo, para somar
\(1\) x \(1\) proceda como a seguir:

\begin{Shaded}
\begin{Highlighting}[]
\DecValTok{1}\SpecialCharTok{*}\DecValTok{1}
\end{Highlighting}
\end{Shaded}

\begin{verbatim}
[1] 1
\end{verbatim}

\subsection{Divisão}\label{divisuxe3o}

Para efetuar uma divisão, você deve utilizar a barra simples
\emph{``}/'' como operador de divisão. Por exemplo, para dividr 4 por 2,
proceda como a seguir:

\begin{Shaded}
\begin{Highlighting}[]
\DecValTok{4}\SpecialCharTok{/}\DecValTok{2}
\end{Highlighting}
\end{Shaded}

\begin{verbatim}
[1] 2
\end{verbatim}

\subsection{Potência}\label{potuxeancia}

Para efetuar uma potenciação, você deve utilizar o circunflexo
\emph{``}\^{}'' ou o duplo asterisco ``**'' como operador de
multiplicação. Por exemplo, para calcular \(2^3\) proceda como a seguir:

\begin{Shaded}
\begin{Highlighting}[]
\DecValTok{2}\SpecialCharTok{\^{}}\DecValTok{3}
\end{Highlighting}
\end{Shaded}

\begin{verbatim}
[1] 8
\end{verbatim}

O que também pode ser feito da seguinte maneira:

\begin{Shaded}
\begin{Highlighting}[]
\DecValTok{2}\SpecialCharTok{**}\DecValTok{3}
\end{Highlighting}
\end{Shaded}

\begin{verbatim}
[1] 8
\end{verbatim}

\subsection{Raíz quadrada}\label{rauxedz-quadrada}

Para efetuar uma radiciação, você deve utilizar o comando \emph{sqrt}
posisionando o número em prêntesis. Por exemplo, para calcular
\(\sqrt{16}\) proceda como a seguir:

\begin{Shaded}
\begin{Highlighting}[]
\FunctionTok{sqrt}\NormalTok{(}\DecValTok{16}\NormalTok{)}
\end{Highlighting}
\end{Shaded}

\begin{verbatim}
[1] 4
\end{verbatim}

\subsection{Logaritmo}\label{logaritmo}

Para calcular o logaritmo de um número, você deve utilizar ocomando
\emph{log10} posisionando o número em prêntesis. Por exemplo, para
calcular \(log(2)\) proceda como a seguir:

\begin{Shaded}
\begin{Highlighting}[]
\FunctionTok{log10}\NormalTok{(}\DecValTok{2}\NormalTok{)}
\end{Highlighting}
\end{Shaded}

\begin{verbatim}
[1] 0.30103
\end{verbatim}

\subsection{Logaritmo natural}\label{logaritmo-natural}

Para calcular o logaritmo natural de um número, você deve utilizar
ocomando \emph{log} posisionando o número em prêntesis. Por exemplo,
para calcular \(ln(2)\) proceda como a seguir:

\begin{Shaded}
\begin{Highlighting}[]
\FunctionTok{log}\NormalTok{(}\DecValTok{2}\NormalTok{)}
\end{Highlighting}
\end{Shaded}

\begin{verbatim}
[1] 0.6931472
\end{verbatim}

\subsection{Seno}\label{seno}

Para calcular o seno de um número, você deve utilizar ocomando
\emph{sin} posisionando o número em prêntesis. Por exemplo, para
calcular o seno de 90 proceda como a seguir:

\begin{Shaded}
\begin{Highlighting}[]
\FunctionTok{sin}\NormalTok{(}\DecValTok{90}\NormalTok{)}
\end{Highlighting}
\end{Shaded}

\begin{verbatim}
[1] 0.8939967
\end{verbatim}

\subsection{Cosseno}\label{cosseno}

Para calcular o cosseno de um número, você deve utilizar ocomando
\emph{cos} posisionando o número em prêntesis. Por exemplo, para
calcular o cosseno de 90 proceda como a seguir:

\begin{Shaded}
\begin{Highlighting}[]
\FunctionTok{cos}\NormalTok{(}\DecValTok{90}\NormalTok{)}
\end{Highlighting}
\end{Shaded}

\begin{verbatim}
[1] -0.4480736
\end{verbatim}

\subsection{Tangente}\label{tangente}

Para calcular a tangente de um número, você deve utilizar ocomando
\emph{tan} posisionando o número em prêntesis. Por exemplo, para
calcular a tangente de 90 proceda como a seguir:

\begin{Shaded}
\begin{Highlighting}[]
\FunctionTok{tan}\NormalTok{(}\DecValTok{90}\NormalTok{)}
\end{Highlighting}
\end{Shaded}

\begin{verbatim}
[1] -1.9952
\end{verbatim}

\subsection{Divisão inteira}\label{divisuxe3o-inteira}

Algumas divisões resultam em números não inteiros, de tal modo que o
resultado é composto por um número inteiro aderido de um ``resto''. Para
calcular a divisão sem o resto você deve usar o operador \emph{\%/\%}.
Por exemplo, uma divisão inteira de cinco por dois deve resultar em dois
e pode ser feita da seguinte maneira:

\begin{Shaded}
\begin{Highlighting}[]
\DecValTok{5} \SpecialCharTok{\%/\%} \DecValTok{2}
\end{Highlighting}
\end{Shaded}

\begin{verbatim}
[1] 2
\end{verbatim}

\subsection{Resto da divisão}\label{resto-da-divisuxe3o}

Porém, se o usuário estiver interessado em obter apenas o resto da
divisão, pode usar o operador \emph{\%\%}. No caso do exemplo anterior,
o resto da divisão é um e pode ser obtido da seguinte maneira:

\begin{Shaded}
\begin{Highlighting}[]
\DecValTok{5} \SpecialCharTok{\%\%} \DecValTok{2}
\end{Highlighting}
\end{Shaded}

\begin{verbatim}
[1] 1
\end{verbatim}

\section{Operadores lógicos}\label{operadores-luxf3gicos}

Os operadores lógicos são usados para comparar valores. O principal
operador lógico de uma linguagem de programação é o operador de
afirmação ou negação. Sempre que houver uma afirmação, a lingaugem
retornará um sinal de verdadeiro (TRUE) e sempre que houver uma negação,
a lingaugem retornará um sinal de falso (FALSE). Nos comandos, o TRUE
pode ser substituído pelo T, enquanto o FALSE pode ser substituído pelo
F. A linguagem R atribui o valor zero para as negações e o valor um para
as afirmações. Assim, \emph{TRUE = 1} e \emph{FALSE = 0} sempre
ocorrerá.

\subsection{Igualdade}\label{igualdade}

Para checar uma condição de igualdade, você deve usar o operador
``\emph{==}''. Por exemplo, para checar se dois é igual a três, você
deve proceder como:

\begin{Shaded}
\begin{Highlighting}[]
\DecValTok{2} \SpecialCharTok{==} \DecValTok{3}
\end{Highlighting}
\end{Shaded}

\begin{verbatim}
[1] FALSE
\end{verbatim}

\subsection{Desigualdade maior que}\label{desigualdade-maior-que}

Para checar uma condição de desigualdade na forma de maior que, isto é,
para verificar se um valor é maior que outro, você deve usar o operador
``\emph{\textgreater{}}''. Por exemplo, para checar se dois é maior que
três, você deve proceder como:

\begin{Shaded}
\begin{Highlighting}[]
\DecValTok{2} \SpecialCharTok{\textgreater{}} \DecValTok{3}
\end{Highlighting}
\end{Shaded}

\begin{verbatim}
[1] FALSE
\end{verbatim}

\subsection{Desigualdade menor que}\label{desigualdade-menor-que}

Para checar uma condição de desigualdade na forma de menor que, isto é,
para verificar se um valor é menor que outro, você deve usar o operador
``\emph{\textless{}}''. Por exemplo, para checar se dois é menor que
três, você deve proceder como:

\begin{Shaded}
\begin{Highlighting}[]
\DecValTok{2} \SpecialCharTok{\textless{}} \DecValTok{3}
\end{Highlighting}
\end{Shaded}

\begin{verbatim}
[1] TRUE
\end{verbatim}

\subsection{Desigualdade maior ou
igual}\label{desigualdade-maior-ou-igual}

Para checar uma condição de desigualdade na forma de maior ou igual a,
isto é, para verificar se um valor é maior ou igual outro, você deve
usar o operador ``\emph{\textgreater=}''. Por exemplo, para checar se
dois é maior ou igual a três, você deve proceder como:

\begin{Shaded}
\begin{Highlighting}[]
\DecValTok{2} \SpecialCharTok{\textgreater{}=} \DecValTok{3}
\end{Highlighting}
\end{Shaded}

\begin{verbatim}
[1] FALSE
\end{verbatim}

\subsection{Desigualdade menor ou
igual}\label{desigualdade-menor-ou-igual}

Para checar uma condição de desigualdade na forma de menor ou igual a,
isto é, para verificar se um valor é menor ou igual outro, você deve
usar o operador ``\emph{\textless=}''. Por exemplo, para checar se dois
é menor ou igual a três, você deve proceder como:

\begin{Shaded}
\begin{Highlighting}[]
\DecValTok{2} \SpecialCharTok{\textless{}=} \DecValTok{3}
\end{Highlighting}
\end{Shaded}

\begin{verbatim}
[1] TRUE
\end{verbatim}

\subsection{Diferente de}\label{diferente-de}

Para checar se um valor é diferente de outro, você deve usar o operador
``!=''. Por exemplo, para checar se dois é diferente de três, você deve
proceder como:

\begin{Shaded}
\begin{Highlighting}[]
\DecValTok{2} \SpecialCharTok{!=} \DecValTok{3}
\end{Highlighting}
\end{Shaded}

\begin{verbatim}
[1] TRUE
\end{verbatim}

\section{Objetos}\label{objetos}

As linguagens de programação geralmente são identificadas ao objeto,
isto é, é possível criar um objeto que representa algum ítem ou valor.
Em R, os objetos devem ser criados om um indicativo de igualdade ``=''.
Por exemplo, imagine que precisamos criar um objeto com o nome ``idade''
contendo a idade de uma pessoa em anos. Este procedimento é feito
informando o comando ``nome do objeto = valor do objeto'' conforme
demonstrado a seguir:

\begin{Shaded}
\begin{Highlighting}[]
\NormalTok{idade }\OtherTok{=} \DecValTok{18}
\end{Highlighting}
\end{Shaded}

Ao executar esse comando, um novo objeto surgirá na aba
\emph{Environment} do painel 3. Esse objeto tem o nome ``idade'' e
recebe um valor de 18. Para visualizar o valor do objeto, o usuário pode
usar a função \emph{print} que imprimirá no painel 2 (console) o valor
referente ao objeto mencionado.

\begin{Shaded}
\begin{Highlighting}[]
\FunctionTok{print}\NormalTok{(idade)}
\end{Highlighting}
\end{Shaded}

\begin{verbatim}
[1] 18
\end{verbatim}

Tendo feito isso, e dado que o objeto de nome \emph{idade} e valor 18
está no \emph{Environment}, cada vez que esse objeto for mencionado a
linguagem R reconhecerá que se trata do número 18. Para exemplificar,
suponha que uma pessoa é considerada idosa a partir dos 60 anos e
suponha que você precise descobrir quantos anos ainda restam para que
essa pessoa com idade = 18 se torne idosa. Nesse caso, você deve
proceder como:

\begin{Shaded}
\begin{Highlighting}[]
\DecValTok{60} \SpecialCharTok{{-}}\NormalTok{ idade}
\end{Highlighting}
\end{Shaded}

\begin{verbatim}
[1] 42
\end{verbatim}

\section{Tipos de objetos}\label{tipos-de-objetos}

Os objetos são maneiras de armazenar informações em um dado arranjo. Em
R, essas informações podem ser arranjadas em funções, vetores, matrizes,
listas, arrays ou quadros de dados (\emph{data frames}). Cada objeto tem
a sua função específica e deve ser usado conforme a necessidade. Por
exemplo, uma matriz é ideal para armazenar objetos com duas dimensões
(linha e coluna) mas não é adequada para agrupar objetos com três
dimensões, nesse caso melhor seria usar um array ou uma lista.

\subsection{Vetores}\label{vetores}

Os vetores são objetos que servem para guardar informações
unidimensionais, isto é, informações que podem ser escritas em uma única
linha ou coluna. Por exemplo, imagine que você trabalhou cinco dias em
um emprego e notou em uma planilha o seu salário de cada dia. Suponha
que os seus ganhos dia após dia em reais foram 50.00, 52.00, 55.00,
48.00, 60.00. Se você anotou essas informações em uma linha de uma
planilha, então você tem um vetor linha. Analogamente, se as informações
foram anotadas em uma coluna de uma planilha, tem-se um vetor coluna. Se
você chamou essa planilha de ``salario'', então isso é o mesmo que:

\[
salario = [50.00, 52.00, 55.00,48.00,60.00]
\]

Para digitar esse vetor em \emph{R} deve-se usar o operador de vetores
\emph{c()}, sempre colocando os valores dentro do parêntesis separando
cada valor por uma vírgula. Lembre-se que o separador decimal da
linguagem \emph{R} é o ponto e que a vírgula é um separador de valores.
Com isso, o vetor anterior deve ser escrito como:

\begin{Shaded}
\begin{Highlighting}[]
\NormalTok{salario }\OtherTok{=} \FunctionTok{c}\NormalTok{(}\FloatTok{50.00}\NormalTok{, }\FloatTok{52.00}\NormalTok{, }\FloatTok{55.00}\NormalTok{, }\FloatTok{48.00}\NormalTok{, }\FloatTok{60.00}\NormalTok{)}
\end{Highlighting}
\end{Shaded}

Feito isso, um objeto de nome \emph{salario} irá aparecer no ambiente de
trabalho. Note que o nome do objeto é sucedido do termo \emph{num
{[}1:5{]}}, isso indica que se trata de um vetor numérico com cinco
elementos. Um detalhe importante a ser mencionado é o fato de que
valores não numéricos também podem ser armazenados em vetores, por
exemplo:

\begin{Shaded}
\begin{Highlighting}[]
\NormalTok{nomes }\OtherTok{=} \FunctionTok{c}\NormalTok{(}\StringTok{"João"}\NormalTok{, }\StringTok{"Maria"}\NormalTok{, }\StringTok{"José"}\NormalTok{)}
\end{Highlighting}
\end{Shaded}

Note que \textbf{os valores não numéricos sempre devem estar entre
aspas}. Para checar se um dado objeto é um vetor, o usuário pode usar a
função \emph{is.vector()} indicando o nome do objeto entre parêntesis.
Por exemplo, para checar se o objeto \emph{salario} é um vetor, proceda
conforme a seguir:

\begin{Shaded}
\begin{Highlighting}[]
\FunctionTok{is.vector}\NormalTok{(salario)}
\end{Highlighting}
\end{Shaded}

\begin{verbatim}
[1] TRUE
\end{verbatim}

Caso o elemento de fato seja um vetor, o output obtido será \emph{TRUE},
do contrário o output será \emph{FALSE}. Para transformar um determinado
objeto em um vetor, o usuário pode usar a função \emph{as.vector(),}
indicando o nome do objeto entre parêntesis. Por exemplo, para
transformar uma sequência de 1 a 10 em um vetor, proceda conforme a
seguir:

\begin{Shaded}
\begin{Highlighting}[]
\NormalTok{sq }\OtherTok{=} \FunctionTok{as.vector}\NormalTok{(}\DecValTok{1}\SpecialCharTok{:}\DecValTok{10}\NormalTok{)}
\end{Highlighting}
\end{Shaded}

\subsection{Matrizes}\label{matrizes}

As matrizes são objetos que servem para guardar informações
bidimensionais, isto é, informações que podem ser escritas em um
múltiplas linhas e múltiplas colunas, desde que o usuário precise
realizar operações algébricas com esses valores. Para exemplificar,
considere o exemplo anterior do salário. Considere agora que você
trabalhou cinco dias da semana não em um mas em dois empregos. Agora
você vai atribuir um diasda semana para cada linha e vai anotar os
ganhos de cada emprego em colunas diferentes. Suponha agora que os
ganhos do emprego 2 foram de 140.00, 160.00, 165.00, 150.00 e 155.00.
Isso equivale a:

\[
salario = \left[
\begin{array}{cc}
Emprego 1 & Emprego 2\\ \hline
50.00 & 140.00\\
52.00 & 160.00\\
55.00 & 165.00\\
48.00 & 150.00\\
60.00 & 155.00 \\
\end{array}
\right]
\]

Agora a planilha de ganhos possui dois vetores coluna de cinco elementos
cada ou cinco vetores linha de dois elementos cada. Para informar essa
planilha como matriz no \emph{R}, o usuário deve usar a função
\emph{matrix}. Nessa função o usuário deve digitar os elementos da
planilha linha por linha em um único vetor e indicar isso com o
parâmetro \emph{by.row = TRUE.} Ou se preferir o usuário pode digitar os
elementos da planilha coluna por coluna em um único vetor e indicar isso
com o parâmetro \emph{by.row = FALSE.} O usuário também deve informar o
número de linhas da matriz com o parâmetro \emph{nrow} e o número de
colunas com o parâmetro \emph{ncol}. Para repassar a planilha anterior
em \emph{R} na forma de matriz, proceda conforme a seguir:

\begin{Shaded}
\begin{Highlighting}[]
\NormalTok{salario }\OtherTok{=} \FunctionTok{matrix}\NormalTok{(}
  \FunctionTok{c}\NormalTok{(}\DecValTok{50}\NormalTok{,}\DecValTok{140}\NormalTok{,}\DecValTok{52}\NormalTok{,}\DecValTok{160}\NormalTok{,}\DecValTok{55}\NormalTok{,}\DecValTok{165}\NormalTok{,}\DecValTok{48}\NormalTok{,}\DecValTok{150}\NormalTok{,}\DecValTok{60}\NormalTok{,}\DecValTok{155}\NormalTok{),}
  \AttributeTok{byrow =} \ConstantTok{TRUE}\NormalTok{,}
  \AttributeTok{ncol =} \DecValTok{2}\NormalTok{,}
  \AttributeTok{nrow =} \DecValTok{5}
\NormalTok{)}

\FunctionTok{print}\NormalTok{(salario)}
\end{Highlighting}
\end{Shaded}

\begin{verbatim}
     [,1] [,2]
[1,]   50  140
[2,]   52  160
[3,]   55  165
[4,]   48  150
[5,]   60  155
\end{verbatim}

Isso é o mesmo que fazer:

\begin{Shaded}
\begin{Highlighting}[]
\NormalTok{salario }\OtherTok{=} \FunctionTok{matrix}\NormalTok{(}
  \FunctionTok{c}\NormalTok{(}\DecValTok{50}\NormalTok{,}\DecValTok{52}\NormalTok{,}\DecValTok{55}\NormalTok{,}\DecValTok{48}\NormalTok{,}\DecValTok{60}\NormalTok{,}\DecValTok{140}\NormalTok{,}\DecValTok{160}\NormalTok{,}\DecValTok{165}\NormalTok{,}\DecValTok{150}\NormalTok{,}\DecValTok{155}\NormalTok{),}
  \AttributeTok{byrow =} \ConstantTok{FALSE}\NormalTok{, }\CommentTok{\# agrupamento por coluna}
  \AttributeTok{ncol =} \DecValTok{2}\NormalTok{,}
  \AttributeTok{nrow =} \DecValTok{5}
\NormalTok{)}

\FunctionTok{print}\NormalTok{(salario)}
\end{Highlighting}
\end{Shaded}

\begin{verbatim}
     [,1] [,2]
[1,]   50  140
[2,]   52  160
[3,]   55  165
[4,]   48  150
[5,]   60  155
\end{verbatim}

Para dar nomes às linhas de uma matriz, use a função \emph{rownames(),}
indicando o nome da matriz entre o parêntesis e informando os nomes das
linhas em um vetor. Por exemplo:

\begin{Shaded}
\begin{Highlighting}[]
\FunctionTok{rownames}\NormalTok{(salario) }\OtherTok{=} \FunctionTok{c}\NormalTok{(}\StringTok{"Seg"}\NormalTok{, }\StringTok{"Ter"}\NormalTok{, }\StringTok{"Quar"}\NormalTok{, }\StringTok{"Qui"}\NormalTok{, }\StringTok{"Sex"}\NormalTok{)}
\FunctionTok{print}\NormalTok{(salario)}
\end{Highlighting}
\end{Shaded}

\begin{verbatim}
     [,1] [,2]
Seg    50  140
Ter    52  160
Quar   55  165
Qui    48  150
Sex    60  155
\end{verbatim}

Para dar nomes às colunas de uma matriz, use a função \emph{colnames(),}
indicando o nome da matriz entre o parêntesis e informando os nomes das
colunas em um vetor. Por exemplo:

\begin{Shaded}
\begin{Highlighting}[]
\FunctionTok{colnames}\NormalTok{(salario) }\OtherTok{=} \FunctionTok{c}\NormalTok{(}\StringTok{"Emprego 1"}\NormalTok{,}\StringTok{"Emprego 2"}\NormalTok{)}
\FunctionTok{print}\NormalTok{(salario)}
\end{Highlighting}
\end{Shaded}

\begin{verbatim}
     Emprego 1 Emprego 2
Seg         50       140
Ter         52       160
Quar        55       165
Qui         48       150
Sex         60       155
\end{verbatim}

Para verificar se um determinado objeto é uma matriz, use a função
\emph{is.matrix(),} indicando o nome do objeto entre o parêntesis, por
exemplo:

\begin{Shaded}
\begin{Highlighting}[]
\FunctionTok{is.matrix}\NormalTok{(salario)}
\end{Highlighting}
\end{Shaded}

\begin{verbatim}
[1] TRUE
\end{verbatim}

Caso o elemento de fato seja uma matriz, o output obtido será
\emph{TRUE}, do contrário o output será \emph{FALSE}. Para transformar
um determinado objeto em uma matriz, o usuário pode usar a função
\emph{as.matrix(),} indicando o nome do objeto entre parêntesis. Por
exemplo, para transformar uma sequência de 1 a 10 em uma matriz, proceda
conforme a seguir:

\begin{Shaded}
\begin{Highlighting}[]
\NormalTok{sq }\OtherTok{=} \FunctionTok{as.matrix}\NormalTok{(}\DecValTok{1}\SpecialCharTok{:}\DecValTok{10}\NormalTok{)}
\FunctionTok{print}\NormalTok{(sq)}
\end{Highlighting}
\end{Shaded}

\begin{verbatim}
      [,1]
 [1,]    1
 [2,]    2
 [3,]    3
 [4,]    4
 [5,]    5
 [6,]    6
 [7,]    7
 [8,]    8
 [9,]    9
[10,]   10
\end{verbatim}

\subsubsection{Operações com
matrizes}\label{operauxe7uxf5es-com-matrizes}

\paragraph{Soma de matrizes}\label{soma-de-matrizes}

A soma de matrizes em \emph{R} não apresenta diferenças das operações
convencionais de soma, ou seja, é feita usando o operador de soma ``+''.
Para exemplificar, considere as dias matrizes a seguir:

\[ matriz1 = \left[ \begin{array}{cc} 0 & 2 \\ 3 & 1 \end{array} \right] \quad \quad \quad \quad matiz2 = \left[ \begin{array}{cc} 5 & 3\\7 & 0 \end{array} \right] \]

Para gerar uma nova matriz de nome \emph{matriz3} contendo a soma da
matriz 1 com a matriz 2, basta proceder conforme a seguir:

\begin{Shaded}
\begin{Highlighting}[]
\NormalTok{matriz1 }\OtherTok{=} \FunctionTok{matrix}\NormalTok{(}\FunctionTok{c}\NormalTok{(}\DecValTok{0}\NormalTok{,}\DecValTok{2}\NormalTok{,}\DecValTok{3}\NormalTok{,}\DecValTok{1}\NormalTok{), }\AttributeTok{nrow =} \DecValTok{2}\NormalTok{, }\AttributeTok{ncol =} \DecValTok{2}\NormalTok{, }\AttributeTok{byrow =} \ConstantTok{TRUE}\NormalTok{)}
\NormalTok{matriz2 }\OtherTok{=} \FunctionTok{matrix}\NormalTok{(}\FunctionTok{c}\NormalTok{(}\DecValTok{5}\NormalTok{,}\DecValTok{3}\NormalTok{,}\DecValTok{7}\NormalTok{,}\DecValTok{0}\NormalTok{), }\AttributeTok{nrow =} \DecValTok{2}\NormalTok{, }\AttributeTok{ncol =} \DecValTok{2}\NormalTok{, }\AttributeTok{byrow =} \ConstantTok{TRUE}\NormalTok{)}
\NormalTok{matriz3 }\OtherTok{=}\NormalTok{ matriz1 }\SpecialCharTok{+}\NormalTok{ matriz2}
\FunctionTok{print}\NormalTok{(matriz3)}
\end{Highlighting}
\end{Shaded}

\begin{verbatim}
     [,1] [,2]
[1,]    5    5
[2,]   10    1
\end{verbatim}

\paragraph{Subtração de matrizes}\label{subtrauxe7uxe3o-de-matrizes}

De maneira análoga à operação de soma, a subtração de matrizes é feita
usando o operador de subtração ``-''. Nesse caso, os elementos de cada
posição das matries são subtraídos. Para exemplificar, considere
subtrair a matriz 1 da matriz 2, procedendo de acordo com o código a
seguir:

\begin{Shaded}
\begin{Highlighting}[]
\NormalTok{matriz3 }\OtherTok{=}\NormalTok{ matriz1 }\SpecialCharTok{{-}}\NormalTok{ matriz2}
\FunctionTok{print}\NormalTok{(matriz3)}
\end{Highlighting}
\end{Shaded}

\begin{verbatim}
     [,1] [,2]
[1,]   -5   -1
[2,]   -4    1
\end{verbatim}

É importante ressaltar que a soma e a subtração de matrizes só podem ser
feitas com matrizes de mesma dimensão.

\paragraph{Multiplicação de
matrizes}\label{multiplicauxe7uxe3o-de-matrizes}

Na multiplicação, não é correto utilizar o asterisco como operador de
multiplicação dado que multiplicar os termos de mesma posição de duas
matrizes não é a maneira correta de efetuar o produto de matrizes. Para
essa tarefa, o operador de multiplicação agora é \(\%*\%\) . Para
exemplificar, considere multiplicarr a matriz 1 pela matriz 2.

\begin{Shaded}
\begin{Highlighting}[]
\NormalTok{matriz3 }\OtherTok{=}\NormalTok{ matriz1 }\SpecialCharTok{\%*\%}\NormalTok{ matriz2}
\FunctionTok{print}\NormalTok{(matriz3)}
\end{Highlighting}
\end{Shaded}

\begin{verbatim}
     [,1] [,2]
[1,]   14    0
[2,]   22    9
\end{verbatim}

\paragraph{Matriz transposta}\label{matriz-transposta}

A transposta de uma matriz nada mais é do que reorganizar as linhas como
colunas e as colunas como linhas. Para obter a transposta de uma matriz
em \emph{R} basta usar a função \emph{t()}, indicando entre parêntesis o
nome da matriz que se deseja transpor. Para exemplificar, considere
transpor a matriz 1. Nesse caso, deve-se proceder conforme a seguir:

\begin{Shaded}
\begin{Highlighting}[]
\FunctionTok{t}\NormalTok{(matriz1)}
\end{Highlighting}
\end{Shaded}

\begin{verbatim}
     [,1] [,2]
[1,]    0    3
[2,]    2    1
\end{verbatim}

\paragraph{Matriz inversa}\label{matriz-inversa}

Uma matriz \(M\) pode ser invertida em \emph{R} para obter \(M^{-1}\)
usando o comando \emph{solve()} e indicando dentro do parêntesis a
matriz que se deseja inverter. Para exemplificar, considere obter a
inversa da matriz 1. Nesse caso, deve-se proceder conforme a seguir:

\begin{Shaded}
\begin{Highlighting}[]
\FunctionTok{solve}\NormalTok{(matriz1)}
\end{Highlighting}
\end{Shaded}

\begin{verbatim}
           [,1]      [,2]
[1,] -0.1666667 0.3333333
[2,]  0.5000000 0.0000000
\end{verbatim}

\paragraph{Exemplo de operações com matrizes: O estimador de mínimos
quadrados
ordinários}\label{exemplo-de-operauxe7uxf5es-com-matrizes-o-estimador-de-muxednimos-quadrados-ordinuxe1rios}

Na estatística, os coeficientes lineares de uma equação linear com
múltiplos argumentos podem ser calculados por meio do método de mínimos
quadrados ordinários. Se essa equação é:

\[ y = \beta_0 + \beta_1x_1 + \beta_2x_2 + \dots + \beta_n x_n \]

Um termo de erro \(\varepsilon\) é adicionado e a relação anterior pode
ser escrita matricialmente como:

\[ \textbf{y} = \textbf{x} \beta + \varepsilon \]

Mininiando a soma dos erros quadráticos, o vetor \(\beta\) de parâmetros
estimados é:

\[ \beta = (\textbf{x}^\prime \textbf{x})^{-1}\textbf{x}^\prime \textbf{y} \]

Para exemplificar, considere usar a base natova sobre automóveis
\emph{mtcars} e suponha que estejamos interessados em saber a relação
entre o consumo do automóvel (mpg) e as variáveis peso (wt) e número de
cilindros (cyl). Nesse caso, a matriz \emph{y} são os valores da coluna
mpg e a matriz x é composta pelas colunas wt e cyl. O vetor de
parâmetros estimados pode ser calculado de acordo com o seguinte
processo:

\begin{Shaded}
\begin{Highlighting}[]
\NormalTok{y }\OtherTok{=} \FunctionTok{matrix}\NormalTok{(mtcars}\SpecialCharTok{$}\NormalTok{mpg, }\AttributeTok{ncol =} \DecValTok{1}\NormalTok{) }\CommentTok{\# matriz y}
\NormalTok{x }\OtherTok{=} \FunctionTok{matrix}\NormalTok{(}\FunctionTok{c}\NormalTok{(}\FunctionTok{rep}\NormalTok{(}\DecValTok{1}\NormalTok{,}\FunctionTok{nrow}\NormalTok{(mtcars)), mtcars}\SpecialCharTok{$}\NormalTok{wt, mtcars}\SpecialCharTok{$}\NormalTok{cyl), }\AttributeTok{ncol =} \DecValTok{3}\NormalTok{) }\CommentTok{\# matriz x}
\NormalTok{tx }\OtherTok{=} \FunctionTok{t}\NormalTok{(x) }\CommentTok{\# transposta da matriz x}
\NormalTok{beta }\OtherTok{=} \FunctionTok{solve}\NormalTok{(tx}\SpecialCharTok{\%*\%}\NormalTok{x)}\SpecialCharTok{\%*\%}\NormalTok{tx}\SpecialCharTok{\%*\%}\NormalTok{y}
\FunctionTok{print}\NormalTok{(beta)}
\end{Highlighting}
\end{Shaded}

\begin{verbatim}
          [,1]
[1,] 39.686261
[2,] -3.190972
[3,] -1.507795
\end{verbatim}

Isso é o mesmo que fazer:

\begin{Shaded}
\begin{Highlighting}[]
\FunctionTok{summary}\NormalTok{(}\FunctionTok{lm}\NormalTok{(mtcars}\SpecialCharTok{$}\NormalTok{mpg }\SpecialCharTok{\textasciitilde{}} \DecValTok{1} \SpecialCharTok{+}\NormalTok{ mtcars}\SpecialCharTok{$}\NormalTok{wt }\SpecialCharTok{+}\NormalTok{ mtcars}\SpecialCharTok{$}\NormalTok{cyl))}
\end{Highlighting}
\end{Shaded}

\begin{verbatim}

Call:
lm(formula = mtcars$mpg ~ 1 + mtcars$wt + mtcars$cyl)

Residuals:
    Min      1Q  Median      3Q     Max 
-4.2893 -1.5512 -0.4684  1.5743  6.1004 

Coefficients:
            Estimate Std. Error t value Pr(>|t|)    
(Intercept)  39.6863     1.7150  23.141  < 2e-16 ***
mtcars$wt    -3.1910     0.7569  -4.216 0.000222 ***
mtcars$cyl   -1.5078     0.4147  -3.636 0.001064 ** 
---
Signif. codes:  0 '***' 0.001 '**' 0.01 '*' 0.05 '.' 0.1 ' ' 1

Residual standard error: 2.568 on 29 degrees of freedom
Multiple R-squared:  0.8302,    Adjusted R-squared:  0.8185 
F-statistic: 70.91 on 2 and 29 DF,  p-value: 6.809e-12
\end{verbatim}

\subsection{Arrays}\label{arrays}

O conceito de array generaliza a idéia de matrix. Enquanto em uma matrix
os elementos são organizados em duas dimensões (linhas e colunas), em um
array os elementos podem ser organizados em um número arbitrário de
dimensões. Em \emph{R} um array é definido utilizando a função array().
O usuário deve informar dentro do parêntesis as informações sempre
ordenadas coluna a coluna e indicar as dimensões do array com o
parâmetro \emph{dim}. Por exemplo, imagine que queiramos armazenar o
consumo de dois bens por duas famílias em três anos, neste caso, teremos
que criar um array de dimensão 2 x 2 x 3. Para exemplificar, imagine que
esse consumo seja o demonstrado a seguir:

\[
\begin{array}{ccc}
Ano 1 = \left[
\begin{array}{c|cc}
& Bem 1 & Bem 2 \\ \hline
Familia 1 & 0 & 5\\
Familia 2 & 2 & 3\\
\end{array}
\right]
&
Ano 2 = \left[
\begin{array}{c|cc}
& Bem 1 & Bem 2 \\ \hline
Familia 1 & 1 & 4\\
Familia 2 & 3 & 2\\
\end{array}
\right]
&\\
Ano 3 = \left[
\begin{array}{c|cc}
& Bem 1 & Bem 2 \\ \hline
Familia 1 & 2 & 3\\
Familia 2 & 2 & 3\\
\end{array}
\right]
\end{array}
\]

Em \emph{R} isso equivale a:

\begin{Shaded}
\begin{Highlighting}[]
\NormalTok{consumo }\OtherTok{=} \FunctionTok{array}\NormalTok{(}
  \FunctionTok{c}\NormalTok{(}
    \DecValTok{0}\NormalTok{,}\DecValTok{2}\NormalTok{,}\DecValTok{5}\NormalTok{,}\DecValTok{3}\NormalTok{, }\CommentTok{\# Ano 1}
    \DecValTok{1}\NormalTok{,}\DecValTok{3}\NormalTok{,}\DecValTok{4}\NormalTok{,}\DecValTok{2}\NormalTok{, }\CommentTok{\# Ano 2}
    \DecValTok{2}\NormalTok{,}\DecValTok{2}\NormalTok{,}\DecValTok{3}\NormalTok{,}\DecValTok{3} \CommentTok{\# ano 3}
\NormalTok{  ),}
  \AttributeTok{dim =} \FunctionTok{c}\NormalTok{(}\DecValTok{2}\NormalTok{,}\DecValTok{2}\NormalTok{,}\DecValTok{3}\NormalTok{)}
\NormalTok{)}
\FunctionTok{print}\NormalTok{(consumo)}
\end{Highlighting}
\end{Shaded}

\begin{verbatim}
, , 1

     [,1] [,2]
[1,]    0    5
[2,]    2    3

, , 2

     [,1] [,2]
[1,]    1    4
[2,]    3    2

, , 3

     [,1] [,2]
[1,]    2    3
[2,]    2    3
\end{verbatim}

Para checar se um objeto é um array, use a função \emph{is.array()}.
Caso o objeto de fato seja um array, o output obtido será \emph{TRUE},
do contrário o output será \emph{FALSE}.

\begin{Shaded}
\begin{Highlighting}[]
\FunctionTok{is.array}\NormalTok{(consumo)}
\end{Highlighting}
\end{Shaded}

\begin{verbatim}
[1] TRUE
\end{verbatim}

\subsection{Listas}\label{listas}

As listas são objetos ideais para guardar informações em múltiplas
dimensões, isto é, informações que possuam múltiplas linhas, múltiplas
colunas e múltiplas planilhas. As listas podem armazenar diversos outros
objetos, incluindo outras listas. Por exemplo, uma lista pode conter um
vetor e uma matriz ou múltiplos vetores e múltiplas matrizes. Para criar
uma lista, o usuário deve usar a função \emph{list()} indicando dentro
do parêntesis os objetos que irão compor a lista. Por exemplo, vamos
criar uma lista contendo os as matrizes de consumo do array anterior e
vamos chamar essa lista de \emph{lista1}.

\begin{Shaded}
\begin{Highlighting}[]
\NormalTok{ano1 }\OtherTok{=} \FunctionTok{matrix}\NormalTok{(}
  \FunctionTok{c}\NormalTok{(}\DecValTok{0}\NormalTok{,}\DecValTok{5}\NormalTok{,}\DecValTok{2}\NormalTok{,}\DecValTok{3}\NormalTok{), }\AttributeTok{byrow =} \ConstantTok{TRUE}\NormalTok{, }\AttributeTok{nrow =} \DecValTok{2}\NormalTok{, }\AttributeTok{ncol =} \DecValTok{2}
\NormalTok{)}
\NormalTok{ano2 }\OtherTok{=} \FunctionTok{matrix}\NormalTok{(}
  \FunctionTok{c}\NormalTok{(}\DecValTok{1}\NormalTok{,}\DecValTok{4}\NormalTok{,}\DecValTok{3}\NormalTok{,}\DecValTok{2}\NormalTok{), }\AttributeTok{byrow =} \ConstantTok{TRUE}\NormalTok{, }\AttributeTok{nrow =} \DecValTok{2}\NormalTok{, }\AttributeTok{ncol =} \DecValTok{2}
\NormalTok{)}
\NormalTok{ano3 }\OtherTok{=} \FunctionTok{matrix}\NormalTok{(}
  \FunctionTok{c}\NormalTok{(}\DecValTok{2}\NormalTok{,}\DecValTok{3}\NormalTok{,}\DecValTok{2}\NormalTok{,}\DecValTok{3}\NormalTok{), }\AttributeTok{byrow =} \ConstantTok{TRUE}\NormalTok{, }\AttributeTok{nrow =} \DecValTok{2}\NormalTok{, }\AttributeTok{ncol =} \DecValTok{2}
\NormalTok{)}

\NormalTok{lista1 }\OtherTok{=} \FunctionTok{list}\NormalTok{(ano1, ano2, ano3)}
\FunctionTok{print}\NormalTok{(lista1)}
\end{Highlighting}
\end{Shaded}

\begin{verbatim}
[[1]]
     [,1] [,2]
[1,]    0    5
[2,]    2    3

[[2]]
     [,1] [,2]
[1,]    1    4
[2,]    3    2

[[3]]
     [,1] [,2]
[1,]    2    3
[2,]    2    3
\end{verbatim}

Para checar se um objeto é uma lista, use a função \emph{is.list()}.
Caso o elemento de fato seja uma lista, o output obtido será
\emph{TRUE}, do contrário o output será \emph{FALSE}.

\begin{Shaded}
\begin{Highlighting}[]
\FunctionTok{is.list}\NormalTok{(lista1)}
\end{Highlighting}
\end{Shaded}

\begin{verbatim}
[1] TRUE
\end{verbatim}

\subsection{Data frames}\label{data-frames}

Os data frames - ou quadro de dados - são objetos que possuem utilidade
semelhante às matrizes, isto é, são ideais para armazenar informações
bidimensionais. No entanto, nos data frames não é possível realizar
operações algébricas como nas matrizes. Os data frames são exatamente
iguais às planilhas do excel, onde cada coluna é uma variável com
valores distribuídos entre as linhas. Essa estrutura de dados é inserida
em \emph{R} usando a função \emph{data.frame(),} de tal modo que o
usuário precisa indicar o nome de cada coluna precedida pelos seus
valores em um vetor. Por exemplo, se usarmos o exemplo anterior do
salário de dois empregos e quisermos colocar as informações dos ganhos
em um data frame de nome \emph{salario\_diario}, devemos proceder
conforme a seguir:

\begin{Shaded}
\begin{Highlighting}[]
\NormalTok{salario\_diario }\OtherTok{=} \FunctionTok{data.frame}\NormalTok{(}
  \AttributeTok{emprego1 =} \FunctionTok{c}\NormalTok{(}\DecValTok{50}\NormalTok{,}\DecValTok{52}\NormalTok{,}\DecValTok{55}\NormalTok{,}\DecValTok{48}\NormalTok{,}\DecValTok{60}\NormalTok{),}
  \AttributeTok{emprego2 =} \FunctionTok{c}\NormalTok{(}\DecValTok{140}\NormalTok{,}\DecValTok{160}\NormalTok{,}\DecValTok{165}\NormalTok{,}\DecValTok{150}\NormalTok{,}\DecValTok{155}\NormalTok{)}
\NormalTok{)}

\FunctionTok{print}\NormalTok{(salario\_diario)}
\end{Highlighting}
\end{Shaded}

\begin{verbatim}
  emprego1 emprego2
1       50      140
2       52      160
3       55      165
4       48      150
5       60      155
\end{verbatim}

Você pode mudar os nomes das colunas dos data frames usando a função
\emph{colnames()}. Por exemplo, se quisermos alterar o nome das colunas
para \emph{trabalho1} e \emph{trabalho2}, devemos proceder conforme a
seguir:

\begin{Shaded}
\begin{Highlighting}[]
\FunctionTok{colnames}\NormalTok{(salario\_diario) }\OtherTok{=} \FunctionTok{c}\NormalTok{(}\StringTok{"trabalho1"}\NormalTok{, }\StringTok{"trabalho2"}\NormalTok{)}
\FunctionTok{print}\NormalTok{(salario\_diario)}
\end{Highlighting}
\end{Shaded}

\begin{verbatim}
  trabalho1 trabalho2
1        50       140
2        52       160
3        55       165
4        48       150
5        60       155
\end{verbatim}

Já os nomes das linhas podem ser alterados por meio da função
\emph{rownames()}.

\begin{Shaded}
\begin{Highlighting}[]
\FunctionTok{rownames}\NormalTok{(salario\_diario) }\OtherTok{=} \FunctionTok{c}\NormalTok{(}\StringTok{"Seg"}\NormalTok{, }\StringTok{"Ter"}\NormalTok{, }\StringTok{"Quar"}\NormalTok{, }\StringTok{"Qui"}\NormalTok{, }\StringTok{"Sex"}\NormalTok{)}
\FunctionTok{print}\NormalTok{(salario\_diario)}
\end{Highlighting}
\end{Shaded}

\begin{verbatim}
     trabalho1 trabalho2
Seg         50       140
Ter         52       160
Quar        55       165
Qui         48       150
Sex         60       155
\end{verbatim}

Para checar se um objeto é um data frame, o usuário deve usar a função
\emph{is.data.frame()} indicando o nome do objeto dentro do parêntesis.
Caso o objeto de fato seja um data frame, o output obtido será
\emph{TRUE}, do contrário o output será \emph{FALSE}.

\begin{Shaded}
\begin{Highlighting}[]
\FunctionTok{is.data.frame}\NormalTok{(salario\_diario)}
\end{Highlighting}
\end{Shaded}

\begin{verbatim}
[1] TRUE
\end{verbatim}

Assim como em uma planilha excel, as colunas dos data frames podem
conter valores numéricos e não numéricos. Maiores detalhes sobre essas
possibilidades de valores serão vistos posteriormente quando abordarmos
as classes dos elementos.

\section{Exercício 1}\label{exercuxedcio-1}

(1) Crie um objeto de nome \emph{valor1} com a operação
\(\left(\frac{5^5}{100}\right)^{0.5}\).

(2) Crie um objeto de nome \emph{valor2} com a operação
\(\left(\frac{3^8}{100}\right)^{0.5}\).

(3) Cheque se
\(\left(\frac{5^5}{100}\right)^{0.5} \neq \sqrt{\left(\frac{5^5}{100}\right)}\).

(4) Cheque se
\(\left(\frac{3^8}{100}\right)^{0.5} \neq \sqrt{\left(\frac{3^8}{100}\right)}\).

(5) Verifique se
\(\left(\frac{5^5}{100}\right)^{0.5}  >= \left(\frac{3^8}{100}\right)^{0.5}\).

(6) Divida \emph{valor1} por \emph{valor2} e verifique se o resultado é
maior que 1.

\section{Exercício 2}\label{exercuxedcio-2}

Considere as seguintes planilhas de dados representando os preços das
ações das empresas Petrobras, Vale e Itau em uma semana de negociações
na bolsa de valores:

\[
petrobras = \left[
\begin{array}{c|ccc}
Dia & Preco & Maximo & Minimo \\ \hline
Seg & 34.5 & 34.75 & 33.8\\
Ter & 34.7 & 35.05 & 34.2\\
Quar & 34.9 & 35.5 & 34.6\\
Qui & 34.55 & 34.9 & 34.3\\
Sex & 34 & 34.55 & 33.6
\end{array}
\right]
\]

\[
vale = \left[
\begin{array}{c|ccc}
Dia & Preco & Maximo & Minimo \\ \hline
Seg & 55.5 & 55.75 & 55.1\\
Ter & 56 & 56.6 & 55.5\\
Quar & 56.5 & 56.75 & 56\\
Qui & 55.8 & 56 & 55.2\\
Sex & 55.2 & 55.8 & 55
\end{array}
\right]
\]

\[
itau = \left[
\begin{array}{c|ccc}
Dia & Preco & Maximo & Minimo \\ \hline
Seg & 28.5 & 28.75 & 28.3\\
Ter & 28.7 & 29.05 & 28.5\\
Quar & 28.9 & 29.2 & 28.7\\
Qui & 28.6 & 28.9 & 28.5\\
Sex & 28.3 & 28.6 & 28.1
\end{array}
\right]
\]

\begin{enumerate}
\def\labelenumi{(\arabic{enumi})}
\tightlist
\item
  Repasse essas três planilhas para o \emph{R} na forma de matrizes,
  nomeando-as de \emph{petrobras}, \emph{vale} e \emph{itau}, assim como
  esboçado na representação das planilhas. Dê nomes às linhas e às
  colunas.
\end{enumerate}

(2) Agora repasse as planilhas para o \emph{R} na forma de data frames,
nomeando-os de \emph{df\_petrobras}, \emph{df\_vale} e \emph{df\_itau}.
Dê nomes às linhas e às colunas.

(3) Transforme as matrizes \emph{petrobras}, \emph{vale} e \emph{itau}
em data frames, nomeando-os de \emph{df\_petrobras\_2},
\emph{df\_vale\_2} e \emph{df\_itau\_2.}

(4) Transforme os data frames \emph{df\_petrobras}, \emph{df\_vale} e
\emph{df\_itau} em matrizes, nomeando-as de \emph{m\_petrobras},
\emph{m\_vale} e \emph{m\_itau.}

(5) Cheque se os objetos \emph{petrobras}, \emph{vale} e \emph{itau} são
matrizes.

(6) Cheque se os objetos \emph{df\_petrobras}, \emph{df\_vale} e
\emph{df\_itau} são data frames.

(7) Repasse as três planilhas para o R em um array nomeando-o de
\emph{preco\_acoes}.

(8) Cheque se o objeto criado na questão anterior é um array.

\bookmarksetup{startatroot}

\chapter{Indexação e operações
indexadas}\label{indexauxe7uxe3o-e-operauxe7uxf5es-indexadas}

Nesse módulo estão expressos os detalhes básicos acerca de operações
fundamentais com diferentes tipos de objetos. O módulo ainda apresenta
os conceitos e as normas básicas sobre posições de elementos em um
objeto (indexação) e também mostra as operações que usam essa indexação
e que podem ser úteis na análise de dados.

\section{Indexação de objetos}\label{indexauxe7uxe3o-de-objetos}

Os objetos são compostos por elementos e esses elementos ocupam uma
posição dentro do objeto. Por exemplo, imagine um vetor
\(x = [Maria, Paulo, Pedro, Ana]\). Esse vetor é composto por quatro
elementos, maria na posição 1, Paulo na posição 2, Pedro na posição 3 e
Ana na posição 4. Agora considere que essas mesmas informações estejam
dispostas em uma matriz.

\[\left[\begin{array}{cc}
Maria & Paulo\\ Pedro & Ana \end{array}
\right]
\]

Agora a posição deve ser visualizada como ``linha por coluna''. Maria
está na linha 1 e coluna 1 (posição {[}\emph{1, 1{]})}, Paulo está na
linha 1 e coluna 2 (posição \emph{{[}1, 2{]})}, Pedro está na linha 2 e
coluna 1 (posição {[}\emph{2, 1{]})} e Ana está na linha 2 e coluna 2
(posição {[}\emph{2, 2{]})}. A essa posição dá-se o nome de indexação. A
indexação nada mais é do que a posição de um elemento em um conjunto,
que nesse caso são os objetos no ambiente de trabalho.

Para verificar qual elemento está em uma determinada posição do objeto,
deve-se informar o nome do objeto precedido da posição entre colchetes,
isto é, \emph{nome do objeto{[}posição na linha, posição na coluna{]}}.
Por exemplo, na matriz anterior, caso queiramos consultar quem está na
linha 1 da coluna 1, devemos proceder conforme a seguir:

\begin{Shaded}
\begin{Highlighting}[]
\NormalTok{nomes }\OtherTok{=} \FunctionTok{matrix}\NormalTok{(}\FunctionTok{c}\NormalTok{(}\StringTok{"Maria"}\NormalTok{, }\StringTok{"Paulo"}\NormalTok{, }\StringTok{"Pedro"}\NormalTok{, }\StringTok{"Ana"}\NormalTok{), }\AttributeTok{byrow =} \ConstantTok{TRUE}\NormalTok{, }\AttributeTok{nrow =} \DecValTok{2}\NormalTok{, }\AttributeTok{ncol =} \DecValTok{2}\NormalTok{)}
\NormalTok{nomes[}\DecValTok{1}\NormalTok{,}\DecValTok{1}\NormalTok{]}
\end{Highlighting}
\end{Shaded}

\begin{verbatim}
[1] "Maria"
\end{verbatim}

A lógica é a mesma em um data frame,

\begin{Shaded}
\begin{Highlighting}[]
\NormalTok{df }\OtherTok{=} \FunctionTok{data.frame}\NormalTok{(}
  \AttributeTok{coluna1 =} \FunctionTok{c}\NormalTok{(}\StringTok{"Maria"}\NormalTok{, }\StringTok{"Pedro"}\NormalTok{),}
  \AttributeTok{coluna2 =} \FunctionTok{c}\NormalTok{(}\StringTok{"Paulo"}\NormalTok{, }\StringTok{"Ana"}\NormalTok{)}
\NormalTok{)}
\NormalTok{df[}\DecValTok{1}\NormalTok{,}\DecValTok{1}\NormalTok{]}
\end{Highlighting}
\end{Shaded}

\begin{verbatim}
[1] "Maria"
\end{verbatim}

Se quisermos nos referir apenas às linhas, então a posição da coluna
deve ficar vazia, isto é, \emph{nome do objeto{[}posição na linha,{]}}.
Seguindo o exemplo anterior, caso queiramos checar quem está na linha 1
da matriz, devemos proceder conforme a seguir:

\begin{Shaded}
\begin{Highlighting}[]
\NormalTok{nomes[}\DecValTok{1}\NormalTok{,]}
\end{Highlighting}
\end{Shaded}

\begin{verbatim}
[1] "Maria" "Paulo"
\end{verbatim}

A mesma lógica se aplica aos data frames.

\begin{Shaded}
\begin{Highlighting}[]
\NormalTok{df[}\DecValTok{1}\NormalTok{,]}
\end{Highlighting}
\end{Shaded}

\begin{verbatim}
  coluna1 coluna2
1   Maria   Paulo
\end{verbatim}

Se quisermos nos referir apenas às colunas, então a posição da linha
deve ficar vazia, isto é, \emph{nome do objeto{[}, posição na
coluna{]}}. Seguindo o exemplo anterior, caso queiramos checar quem está
na coluna 1 da matriz, devemos proceder conforme a seguir:

\begin{Shaded}
\begin{Highlighting}[]
\NormalTok{nomes[,}\DecValTok{1}\NormalTok{]}
\end{Highlighting}
\end{Shaded}

\begin{verbatim}
[1] "Maria" "Pedro"
\end{verbatim}

A mesma lógica se aplica aos data frames.

\begin{Shaded}
\begin{Highlighting}[]
\NormalTok{df[,}\DecValTok{1}\NormalTok{]}
\end{Highlighting}
\end{Shaded}

\begin{verbatim}
[1] "Maria" "Pedro"
\end{verbatim}

A posição das linhas e colunas também pode ser referenciada de acordo
com os nomes. Nesse caso, o procedimento é \emph{nome do objeto{[}``nome
da linha'', ``nome da coluna''{]}}. Para exemplificar, vamos dar nomes
às linhas e as colunas da matriz e do data frame usados nos exemplos
anteriores.

\begin{Shaded}
\begin{Highlighting}[]
\FunctionTok{rownames}\NormalTok{(nomes) }\OtherTok{=} \FunctionTok{c}\NormalTok{(}\StringTok{"linha 1"}\NormalTok{, }\StringTok{"linha 2"}\NormalTok{)}
\FunctionTok{colnames}\NormalTok{(nomes) }\OtherTok{=} \FunctionTok{c}\NormalTok{(}\StringTok{"coluna 1"}\NormalTok{, }\StringTok{"coluna 2"}\NormalTok{)}
\FunctionTok{print}\NormalTok{(nomes)}
\end{Highlighting}
\end{Shaded}

\begin{verbatim}
        coluna 1 coluna 2
linha 1 "Maria"  "Paulo" 
linha 2 "Pedro"  "Ana"   
\end{verbatim}

\begin{Shaded}
\begin{Highlighting}[]
\FunctionTok{rownames}\NormalTok{(df) }\OtherTok{=} \FunctionTok{c}\NormalTok{(}\StringTok{"linha 1"}\NormalTok{, }\StringTok{"linha 2"}\NormalTok{)}
\FunctionTok{print}\NormalTok{(nomes)}
\end{Highlighting}
\end{Shaded}

\begin{verbatim}
        coluna 1 coluna 2
linha 1 "Maria"  "Paulo" 
linha 2 "Pedro"  "Ana"   
\end{verbatim}

Para checar quem está na linha 1, proceda conforme a seguir:

\begin{Shaded}
\begin{Highlighting}[]
\NormalTok{nomes[}\StringTok{"linha 1"}\NormalTok{,]}
\end{Highlighting}
\end{Shaded}

\begin{verbatim}
coluna 1 coluna 2 
 "Maria"  "Paulo" 
\end{verbatim}

E no data frame:

\begin{Shaded}
\begin{Highlighting}[]
\NormalTok{df[}\StringTok{"linha 1"}\NormalTok{, ]}
\end{Highlighting}
\end{Shaded}

\begin{verbatim}
        coluna1 coluna2
linha 1   Maria   Paulo
\end{verbatim}

Para checar quem está na coluna 1, proceda conforme a seguir:

\begin{Shaded}
\begin{Highlighting}[]
\NormalTok{nomes[, }\StringTok{"coluna 1"}\NormalTok{]}
\end{Highlighting}
\end{Shaded}

\begin{verbatim}
linha 1 linha 2 
"Maria" "Pedro" 
\end{verbatim}

E no data frame:

\begin{Shaded}
\begin{Highlighting}[]
\NormalTok{df[, }\StringTok{"coluna1"}\NormalTok{]}
\end{Highlighting}
\end{Shaded}

\begin{verbatim}
[1] "Maria" "Pedro"
\end{verbatim}

Nos data frames, a indexação das colunas pode ser mais simplificada
devido a possibilidade de referenciar as colunas plo nome usando o
cifrão ``\$''. Nesse caso, o procedimento a se fazer é \emph{nome do
objeto\$nome da coluna}. Por exemplo, para se referir a coluna 1 do data
frame anterior, deve-se proceder conforme a seguir:

\begin{Shaded}
\begin{Highlighting}[]
\NormalTok{df}\SpecialCharTok{$}\NormalTok{coluna1}
\end{Highlighting}
\end{Shaded}

\begin{verbatim}
[1] "Maria" "Pedro"
\end{verbatim}

\section{Adicionando linhas a um data
frame}\label{adicionando-linhas-a-um-data-frame}

Tendo comopreendido como funciona a indexação, o próximo passo é
aprender como essa indexação pode ser usada para adicionar, excluir ou
alterar elementos de um objeto. Caso a intenção seja adicionar uma linha
em um data frame, então o procedimento a se fazer é \emph{nome do
objeto{[}número de linhas do objeto + 1,{]}}. Para exemplificar, vamos
incluir uma linha na data frame de nomes contendo os valores
\emph{{[}João, Clara{]}}. Esse procedimento deve ser feito conforme
especificado a seguir:

\begin{Shaded}
\begin{Highlighting}[]
\NormalTok{df[}\DecValTok{3}\NormalTok{,] }\OtherTok{=} \FunctionTok{c}\NormalTok{(}\StringTok{"João"}\NormalTok{, }\StringTok{"Clara"}\NormalTok{)}
\FunctionTok{print}\NormalTok{(df)}
\end{Highlighting}
\end{Shaded}

\begin{verbatim}
        coluna1 coluna2
linha 1   Maria   Paulo
linha 2   Pedro     Ana
3          João   Clara
\end{verbatim}

\section{Adicionando linhas em uma
matriz}\label{adicionando-linhas-em-uma-matriz}

Para adicionar linhas em uma matriz é preciso usar a função
\emph{rbind()}. Essa função também funciona perfeitamente para data
frames. Nesse caso, o procedimento a se fazer é \emph{rbind(nome do
objeto, vetor com os valores da linha adicionada)}. Por exemplo, para
adicionar uma linha com os valores \emph{{[}João, Clara{]}} no data
frame de nomes, deve-se proceder conforme a seguir:

\begin{Shaded}
\begin{Highlighting}[]
\NormalTok{nomes }\OtherTok{=} \FunctionTok{rbind}\NormalTok{(nomes, }\FunctionTok{c}\NormalTok{(}\StringTok{"João"}\NormalTok{, }\StringTok{"Clara"}\NormalTok{))}
\FunctionTok{print}\NormalTok{(nomes)}
\end{Highlighting}
\end{Shaded}

\begin{verbatim}
        coluna 1 coluna 2
linha 1 "Maria"  "Paulo" 
linha 2 "Pedro"  "Ana"   
        "João"   "Clara" 
\end{verbatim}

A mesma lógica se aplica aos data frames:

\begin{Shaded}
\begin{Highlighting}[]
\FunctionTok{rbind}\NormalTok{(df, }\FunctionTok{c}\NormalTok{(}\StringTok{"João"}\NormalTok{, }\StringTok{"Clara"}\NormalTok{))}
\end{Highlighting}
\end{Shaded}

\begin{verbatim}
        coluna1 coluna2
linha 1   Maria   Paulo
linha 2   Pedro     Ana
3          João   Clara
4          João   Clara
\end{verbatim}

\section{Removendo linhas de um
objeto}\label{removendo-linhas-de-um-objeto}

A remoção de linhas de um objeto é feita adicionando um sinal de menos
antes do número da linha indicado em colchetes. Por exemplo, caso
queiramos remover a linha 3 da matriz de nomes, devemos proceder
conforme a seguir:

\begin{Shaded}
\begin{Highlighting}[]
\NormalTok{nomes[}\SpecialCharTok{{-}}\DecValTok{3}\NormalTok{,]}
\end{Highlighting}
\end{Shaded}

\begin{verbatim}
        coluna 1 coluna 2
linha 1 "Maria"  "Paulo" 
linha 2 "Pedro"  "Ana"   
\end{verbatim}

A mesma lógica se aplica aos data frames:

\begin{Shaded}
\begin{Highlighting}[]
\NormalTok{df[}\SpecialCharTok{{-}}\DecValTok{3}\NormalTok{,]}
\end{Highlighting}
\end{Shaded}

\begin{verbatim}
        coluna1 coluna2
linha 1   Maria   Paulo
linha 2   Pedro     Ana
\end{verbatim}

Caso a intenção seja remover múltiplas linhas, então o número das linhas
que serão removidas deve ser indicado em um vetor precedido do sinal de
menos dentro do colchetes. Por exemplo, caso queiramos remover as linhas
2 e 3 da matriz de nomes, devemos proceder conforme a seguir:

\begin{Shaded}
\begin{Highlighting}[]
\NormalTok{nomes[}\SpecialCharTok{{-}}\FunctionTok{c}\NormalTok{(}\DecValTok{2}\NormalTok{,}\DecValTok{3}\NormalTok{),]}
\end{Highlighting}
\end{Shaded}

\begin{verbatim}
coluna 1 coluna 2 
 "Maria"  "Paulo" 
\end{verbatim}

A mesma lógica se aplica aos data frames:

\begin{Shaded}
\begin{Highlighting}[]
\NormalTok{df[}\SpecialCharTok{{-}}\FunctionTok{c}\NormalTok{(}\DecValTok{2}\NormalTok{,}\DecValTok{3}\NormalTok{),]}
\end{Highlighting}
\end{Shaded}

\begin{verbatim}
        coluna1 coluna2
linha 1   Maria   Paulo
\end{verbatim}

\section{Adicionando colunas em um data
frame}\label{adicionando-colunas-em-um-data-frame}

Existem várias maneiras de adicionar colunas em um data frame. O usuário
pode indicar a posição da coluna entre colchetes e em seguida indicar os
valores, pode indicar o nome da coluna em colchetes e em seguida indicar
os valores, pode usar o sifrão para indicar o nome da coluna criada,
\ldots{}

Para exemplificar, suponha que queiramos adicionar uma coluna com os
valores *{[}pessoa 1, pessoa 2, pessoa 3{]} no dtaframe df. Essa tarefa
pode ser feita das seguintes maneiras:

\begin{itemize}
\tightlist
\item
  Indicando a posição em colchetes
\end{itemize}

\begin{Shaded}
\begin{Highlighting}[]
\NormalTok{df[,}\DecValTok{3}\NormalTok{] }\OtherTok{=} \FunctionTok{c}\NormalTok{(}\StringTok{"pessoa 1"}\NormalTok{, }\StringTok{"pessoa 2"}\NormalTok{, }\StringTok{"pessoa 3"}\NormalTok{)}
\FunctionTok{print}\NormalTok{(df)}
\end{Highlighting}
\end{Shaded}

\begin{verbatim}
        coluna1 coluna2       V3
linha 1   Maria   Paulo pessoa 1
linha 2   Pedro     Ana pessoa 2
3          João   Clara pessoa 3
\end{verbatim}

\begin{itemize}
\tightlist
\item
  Indicando o nome da coluna no colchetes.
\end{itemize}

\begin{Shaded}
\begin{Highlighting}[]
\NormalTok{df[, }\StringTok{"coluna3"}\NormalTok{] }\OtherTok{=} \FunctionTok{c}\NormalTok{(}\StringTok{"pessoa 1"}\NormalTok{, }\StringTok{"pessoa 2"}\NormalTok{, }\StringTok{"pessoa 3"}\NormalTok{)}
\FunctionTok{print}\NormalTok{(df)}
\end{Highlighting}
\end{Shaded}

\begin{verbatim}
        coluna1 coluna2  coluna3
linha 1   Maria   Paulo pessoa 1
linha 2   Pedro     Ana pessoa 2
3          João   Clara pessoa 3
\end{verbatim}

\begin{Shaded}
\begin{Highlighting}[]
\NormalTok{df}\OtherTok{=}\NormalTok{df[,}\SpecialCharTok{{-}}\DecValTok{3}\NormalTok{]}
\end{Highlighting}
\end{Shaded}

\begin{itemize}
\tightlist
\item
  Usando o cifrão.
\end{itemize}

\begin{Shaded}
\begin{Highlighting}[]
\NormalTok{df}\SpecialCharTok{$}\NormalTok{coluna3 }\OtherTok{=} \FunctionTok{c}\NormalTok{(}\StringTok{"pessoa 1"}\NormalTok{, }\StringTok{"pessoa 2"}\NormalTok{, }\StringTok{"pessoa 3"}\NormalTok{)}
\FunctionTok{print}\NormalTok{(df)}
\end{Highlighting}
\end{Shaded}

\begin{verbatim}
        coluna1 coluna2  coluna3
linha 1   Maria   Paulo pessoa 1
linha 2   Pedro     Ana pessoa 2
3          João   Clara pessoa 3
\end{verbatim}

\section{Adicionando colunas em uma
matriz}\label{adicionando-colunas-em-uma-matriz}

Para adicionar colunas em uma matriz é preciso usar a função
\emph{cbind()}. Essa função tem a mesma lógica de uso da função
\emph{rbind()} apresentada anteriormente, com a diferença de que a
função \emph{cbind()} posiciona os novos elementos em uma coluna em vez
de uma linha. Para exemplificar, vamos adicionar a mesma coluna incluída
no data frame \emph{df} na matriz de nomes.

\begin{Shaded}
\begin{Highlighting}[]
\NormalTok{nomes }\OtherTok{=} \FunctionTok{cbind}\NormalTok{(nomes, }\FunctionTok{c}\NormalTok{(}\StringTok{"pessoa 1"}\NormalTok{, }\StringTok{"pessoa 2"}\NormalTok{, }\StringTok{"pessoa 3"}\NormalTok{))}
\FunctionTok{print}\NormalTok{(nomes)}
\end{Highlighting}
\end{Shaded}

\begin{verbatim}
        coluna 1 coluna 2           
linha 1 "Maria"  "Paulo"  "pessoa 1"
linha 2 "Pedro"  "Ana"    "pessoa 2"
        "João"   "Clara"  "pessoa 3"
\end{verbatim}

A mesma lógica pode ser aplicada aos data frames fazendo \emph{df =
cbind(df, c(``pessoa 1'', ``pessoa 2'', ``pessoa 3''))}.

\section{Renomeando colunas e linhas
específicas}\label{renomeando-colunas-e-linhas-especuxedficas}

Como visto no capítulo anterior, a função \emph{colnames()} pode ser
usada para renomear colunas de uma matriz ou data frame. Porém, caso o
objetivo seja renomear uma única coluna específica é possível usar a
indexação para realizar essa tarefa. Por exemplo, considere renomear
apenas a coluna 3 do data frame de nomes atribuindo a essa coluna o nome
\emph{``coluna3''}.

\begin{Shaded}
\begin{Highlighting}[]
\FunctionTok{colnames}\NormalTok{(nomes)[}\DecValTok{3}\NormalTok{] }\OtherTok{=} \StringTok{"Coluna3"}
\FunctionTok{print}\NormalTok{(nomes)}
\end{Highlighting}
\end{Shaded}

\begin{verbatim}
        coluna 1 coluna 2 Coluna3   
linha 1 "Maria"  "Paulo"  "pessoa 1"
linha 2 "Pedro"  "Ana"    "pessoa 2"
        "João"   "Clara"  "pessoa 3"
\end{verbatim}

O mesmo pode ser feito para as linhas. Por exemplo, caso queiramos
renomear apenas a linha 3 do data frame de nomes, podemos proceder
conforme a seguir:

\begin{Shaded}
\begin{Highlighting}[]
\FunctionTok{rownames}\NormalTok{(nomes)[}\DecValTok{3}\NormalTok{] }\OtherTok{=} \StringTok{"linha 3"}
\FunctionTok{print}\NormalTok{(nomes)}
\end{Highlighting}
\end{Shaded}

\begin{verbatim}
        coluna 1 coluna 2 Coluna3   
linha 1 "Maria"  "Paulo"  "pessoa 1"
linha 2 "Pedro"  "Ana"    "pessoa 2"
linha 3 "João"   "Clara"  "pessoa 3"
\end{verbatim}

\section{Operações com colunas}\label{operauxe7uxf5es-com-colunas}

Em um data frame novas colunas podem ser geradas por meio de operações
com colunas existentes. Para exemplificar, considere um dataframe com
informações sobre medidas de alunos de uma academia.

\begin{Shaded}
\begin{Highlighting}[]
\NormalTok{alunos }\OtherTok{=} \FunctionTok{data.frame}\NormalTok{(}
\AttributeTok{nome =} \FunctionTok{c}\NormalTok{(}\StringTok{"Aluno 1"}\NormalTok{, }\StringTok{"Aluno 2"}\NormalTok{, }\StringTok{"Aluno 3"}\NormalTok{),}
\AttributeTok{peso =} \FunctionTok{c}\NormalTok{(}\DecValTok{65}\NormalTok{, }\DecValTok{70}\NormalTok{, }\DecValTok{90}\NormalTok{),}
\AttributeTok{altura =} \FunctionTok{c}\NormalTok{(}\FloatTok{1.60}\NormalTok{, }\FloatTok{1.70}\NormalTok{, }\FloatTok{1.78}\NormalTok{)}
\NormalTok{)}
\FunctionTok{print}\NormalTok{(alunos)}
\end{Highlighting}
\end{Shaded}

\begin{verbatim}
     nome peso altura
1 Aluno 1   65   1.60
2 Aluno 2   70   1.70
3 Aluno 3   90   1.78
\end{verbatim}

Imagine que seja necessário adicionar nesse data frame uma nova coluna
com o índice de massa corporal dos alunos (IMC). Sabe-se que:

\[
IMC = \frac{Peso}{Altura^2}
\]

A nova coluna pode ser adicionada indexando a sua posição:

\begin{Shaded}
\begin{Highlighting}[]
\NormalTok{alunos[,}\DecValTok{4}\NormalTok{] }\OtherTok{=}\NormalTok{ alunos[,}\DecValTok{2}\NormalTok{]}\SpecialCharTok{/}\NormalTok{(alunos[,}\DecValTok{3}\NormalTok{]}\SpecialCharTok{\^{}}\DecValTok{2}\NormalTok{)}
\FunctionTok{print}\NormalTok{(alunos)}
\end{Highlighting}
\end{Shaded}

\begin{verbatim}
     nome peso altura       V4
1 Aluno 1   65   1.60 25.39062
2 Aluno 2   70   1.70 24.22145
3 Aluno 3   90   1.78 28.40550
\end{verbatim}

Indexando o nome da coluna:

\begin{Shaded}
\begin{Highlighting}[]
\NormalTok{alunos[}\StringTok{"imc"}\NormalTok{] }\OtherTok{=}\NormalTok{ alunos[}\StringTok{"peso"}\NormalTok{]}\SpecialCharTok{/}\NormalTok{(alunos[}\StringTok{"altura"}\NormalTok{]}\SpecialCharTok{\^{}}\DecValTok{2}\NormalTok{)}
\FunctionTok{print}\NormalTok{(alunos)}
\end{Highlighting}
\end{Shaded}

\begin{verbatim}
     nome peso altura      imc
1 Aluno 1   65   1.60 25.39062
2 Aluno 2   70   1.70 24.22145
3 Aluno 3   90   1.78 28.40550
\end{verbatim}

Ou utilizando o cifrão:

\begin{Shaded}
\begin{Highlighting}[]
\NormalTok{alunos}\SpecialCharTok{$}\NormalTok{imc }\OtherTok{=}\NormalTok{ alunos}\SpecialCharTok{$}\NormalTok{peso}\SpecialCharTok{/}\NormalTok{(alunos}\SpecialCharTok{$}\NormalTok{altura}\SpecialCharTok{\^{}}\DecValTok{2}\NormalTok{)}
\FunctionTok{print}\NormalTok{(alunos)}
\end{Highlighting}
\end{Shaded}

\begin{verbatim}
     nome peso altura      imc
1 Aluno 1   65   1.60 25.39062
2 Aluno 2   70   1.70 24.22145
3 Aluno 3   90   1.78 28.40550
\end{verbatim}

\subsection{Criando uma nova coluna como recorte de uma coluna
existente}\label{criando-uma-nova-coluna-como-recorte-de-uma-coluna-existente}

Imagine o caso em que seja preciso criar uma coluna como um recorte de
valores de uma coluna existente em um data frame. Falando com outras
palavras, imagine que seja preciso adicionar uma coluna contendo uma
parte dos valores contidos em outra coluna. Por exemplo, no data frame
de alunos da academia, imagine que seja preciso mostrar apenas o número
do aluno em uma nova coluna, isto é, em vez de mostrar o termo ``Aluno
1'' suponha que seja preciso mostrar apenas o número ``1''.

Esse procedimento pode ser feito usando a função \emph{substr}. Essa
função é nativa da linguagem \emph{R} e serve para desmembrar valores de
acordo com a posição dos caracteres desses valores. Exemplificando, o
termo ``Aluno 1'' tem sete caracteres que correspondem a seis letras, um
espaço e um número. O número ocupa a sétima posição no termo. Na função
substr temos que indicar a posição inicial e a posição final do conjunto
de caracteres que queremos desmembrar do valor objetivo. A maneira
correta de usar essa função é fazendo \emph{substr(nome do elemento,
posição inicial, posição final)}. Por exemplo, se fizermos
\emph{substr(``Aluno 1'', 1, 2)} o resultado será \emph{``Al}'' que
corresponde aos dois primeiros caracteres do termo utilizado.

No nosso exemplo como o número do aluno inicia e termina na sétima
posição, então o correto a se fazer é:

\begin{Shaded}
\begin{Highlighting}[]
\NormalTok{alunos}\SpecialCharTok{$}\NormalTok{num\_aluno }\OtherTok{=} \FunctionTok{substr}\NormalTok{(alunos}\SpecialCharTok{$}\NormalTok{nome, }\DecValTok{7}\NormalTok{, }\DecValTok{7}\NormalTok{)}
\FunctionTok{print}\NormalTok{(alunos)}
\end{Highlighting}
\end{Shaded}

\begin{verbatim}
     nome peso altura      imc num_aluno
1 Aluno 1   65   1.60 25.39062         1
2 Aluno 2   70   1.70 24.22145         2
3 Aluno 3   90   1.78 28.40550         3
\end{verbatim}

\subsection{Mesclando colunas}\label{mesclando-colunas}

Em vez de separar valores de uma coluna, imagine o caso em que seja
preciso juntar valores. Em \emph{R} isso pode ser facilmente feito
usando a função \emph{paste()}. Essa função junta múltiplos valores
separando-os com um searador indicado pelo usuário. Para usar a função,
basta indicar entre parênteses os valores que serão unificados e indicar
o separador com o parâmetro \emph{``sep''}. Por exemplo:

\begin{Shaded}
\begin{Highlighting}[]
\NormalTok{nome }\OtherTok{=} \StringTok{"João"}
\NormalTok{sobrenome }\OtherTok{=} \StringTok{"Silva"}
\NormalTok{nome\_completo }\OtherTok{=} \FunctionTok{paste}\NormalTok{(nome, sobrenome, }\AttributeTok{sep =} \StringTok{" "}\NormalTok{)}
\FunctionTok{print}\NormalTok{(nome\_completo)}
\end{Highlighting}
\end{Shaded}

\begin{verbatim}
[1] "João Silva"
\end{verbatim}

Essa função pode ser usada para unir colunas. Para exemploficar,
considere o novo data frame com as características dos alunos de uma
academia indicado a seguir:

\begin{Shaded}
\begin{Highlighting}[]
\NormalTok{alunos2 }\OtherTok{=}\NormalTok{ alunos}
\NormalTok{alunos2}\SpecialCharTok{$}\NormalTok{nome }\OtherTok{=} \FunctionTok{c}\NormalTok{(}\StringTok{"João"}\NormalTok{, }\StringTok{"Maria"}\NormalTok{, }\StringTok{"João"}\NormalTok{)}
\end{Highlighting}
\end{Shaded}

E imagine que seja preciso criar um código do aluno, unindo o nome e o
número do aluno separando o nome e um número por um traço. Esse
procedimento pode ser feito conforme indicado a seguir:

\begin{Shaded}
\begin{Highlighting}[]
\NormalTok{alunos2}\SpecialCharTok{$}\NormalTok{cod\_aluno }\OtherTok{=} \FunctionTok{paste}\NormalTok{(alunos2}\SpecialCharTok{$}\NormalTok{nome, alunos2}\SpecialCharTok{$}\NormalTok{num\_aluno, }\AttributeTok{sep =} \StringTok{"{-}"}\NormalTok{)}
\FunctionTok{print}\NormalTok{(alunos2)}
\end{Highlighting}
\end{Shaded}

\begin{verbatim}
   nome peso altura      imc num_aluno cod_aluno
1  João   65   1.60 25.39062         1    João-1
2 Maria   70   1.70 24.22145         2   Maria-2
3  João   90   1.78 28.40550         3    João-3
\end{verbatim}

\section{Reposicionando linhas e
colunas}\label{reposicionando-linhas-e-colunas}

Conhecendo a posição de cada linha e cada coluna, é possível
reordená-las de acordo com as necessidades ou preferências do usuário
usando operações indexadas. Por exemplo, suponha que seja preciso
reordenar as linhas do data frame de alunos para posicionar o aluno 3 na
primeira linha, o aluno 2 na segunda linha e o aluno 1 na terceira
linha. Esse procedimento pode ser feito conforme a seguir:

\begin{Shaded}
\begin{Highlighting}[]
\NormalTok{alunos }\OtherTok{=}\NormalTok{ alunos[}\FunctionTok{c}\NormalTok{(}\DecValTok{3}\NormalTok{,}\DecValTok{2}\NormalTok{,}\DecValTok{1}\NormalTok{),]}
\FunctionTok{print}\NormalTok{(alunos)}
\end{Highlighting}
\end{Shaded}

\begin{verbatim}
     nome peso altura      imc num_aluno
3 Aluno 3   90   1.78 28.40550         3
2 Aluno 2   70   1.70 24.22145         2
1 Aluno 1   65   1.60 25.39062         1
\end{verbatim}

Agora suponha que seja necessário posicionar o nome do aluno na primeira
coluna, a altura na segunda coluna, o peso na terceira coluna e o imc na
última coluna, isto é, suponha que seja preciso inverter a posição do
peso e da altura no data frame. Esse procedimento pode ser feito
conforme indicado a seguir:

\begin{Shaded}
\begin{Highlighting}[]
\NormalTok{alunos }\OtherTok{=}\NormalTok{ alunos[, }\FunctionTok{c}\NormalTok{(}\DecValTok{1}\NormalTok{,}\DecValTok{3}\NormalTok{,}\DecValTok{2}\NormalTok{,}\DecValTok{4}\NormalTok{)]}
\FunctionTok{print}\NormalTok{(alunos)}
\end{Highlighting}
\end{Shaded}

\begin{verbatim}
     nome altura peso      imc
3 Aluno 3   1.78   90 28.40550
2 Aluno 2   1.70   70 24.22145
1 Aluno 1   1.60   65 25.39062
\end{verbatim}

\section{Operações básicas}\label{operauxe7uxf5es-buxe1sicas}

Novas linhas e colunas podem ser adicionadas em um data frame usando
operações básicas como soma, produto, ou estatísticas básicas. Para
demonstrar, considere um data frame contendo a idade de cinco pessoas
conforme a seguir:

\begin{Shaded}
\begin{Highlighting}[]
\NormalTok{df }\OtherTok{=} \FunctionTok{data.frame}\NormalTok{(}
  \AttributeTok{pessoa =} \FunctionTok{c}\NormalTok{(}\StringTok{"pessoa1"}\NormalTok{, }\StringTok{"pessoa2"}\NormalTok{, }\StringTok{"pessoa3"}\NormalTok{, }\StringTok{"pessoa4"}\NormalTok{, }\StringTok{"pessoa"}\NormalTok{),}
  \AttributeTok{idade =} \FunctionTok{c}\NormalTok{(}\DecValTok{25}\NormalTok{, }\DecValTok{50}\NormalTok{, }\DecValTok{68}\NormalTok{,}\DecValTok{45}\NormalTok{,}\ConstantTok{NA}\NormalTok{)}
\NormalTok{)}
\FunctionTok{print}\NormalTok{(df)}
\end{Highlighting}
\end{Shaded}

\begin{verbatim}
   pessoa idade
1 pessoa1    25
2 pessoa2    50
3 pessoa3    68
4 pessoa4    45
5  pessoa    NA
\end{verbatim}

Note que há um valor faltante, indicado pelo termo \emph{NA} que
representa a sigla do ``não disponível'' (do inglês \emph{not
available}). Qualquer operação contendo essa coluna deve indicar que
esse valor ausente deve ser ignorado. Isso pode ser feito adicionando a
opção \emph{na.rm = TRUE}.

\subsection{Soma total}\label{soma-total}

Suponha que queiramos encontrar o somatório da idade de todas as pessoas
contidas no data frame. Isso pode ser facilmente executado usando a
função \emph{sum()}, conforme indicado a seguir:

\begin{Shaded}
\begin{Highlighting}[]
\FunctionTok{sum}\NormalTok{(df}\SpecialCharTok{$}\NormalTok{idade, }\AttributeTok{na.rm =} \ConstantTok{TRUE}\NormalTok{)}
\end{Highlighting}
\end{Shaded}

\begin{verbatim}
[1] 188
\end{verbatim}

\subsection{Produto total}\label{produto-total}

Caso o objetivo seja encontrar o produto de todos os valores de uma dada
coluna, então o ideal é usar a função \emph{prod()}, conforme
demonstrado a seguir:

\begin{Shaded}
\begin{Highlighting}[]
\FunctionTok{prod}\NormalTok{(df}\SpecialCharTok{$}\NormalTok{idade, }\AttributeTok{na.rm =} \ConstantTok{TRUE}\NormalTok{)}
\end{Highlighting}
\end{Shaded}

\begin{verbatim}
[1] 3825000
\end{verbatim}

\subsection{Média}\label{muxe9dia}

Já a média pode ser obtida com a função \emph{mean()}, conforme
demonstrado a seguir:

\begin{Shaded}
\begin{Highlighting}[]
\FunctionTok{mean}\NormalTok{(df}\SpecialCharTok{$}\NormalTok{idade, }\AttributeTok{na.rm =} \ConstantTok{TRUE}\NormalTok{)}
\end{Highlighting}
\end{Shaded}

\begin{verbatim}
[1] 47
\end{verbatim}

\subsection{Mínimo}\label{muxednimo}

O valor mínimo de um dado objeto pode ser computado por meio da função
\emph{min()}, conforme demonstrado a seguir:

\begin{Shaded}
\begin{Highlighting}[]
\FunctionTok{min}\NormalTok{(df}\SpecialCharTok{$}\NormalTok{idade, }\AttributeTok{na.rm =} \ConstantTok{TRUE}\NormalTok{)}
\end{Highlighting}
\end{Shaded}

\begin{verbatim}
[1] 25
\end{verbatim}

\subsection{Máximo}\label{muxe1ximo}

O valor máximo de um dado objeto pode ser computado por meio da função
\emph{max()}, conforme demonstrado a seguir:

\begin{Shaded}
\begin{Highlighting}[]
\FunctionTok{max}\NormalTok{(df}\SpecialCharTok{$}\NormalTok{idade, }\AttributeTok{na.rm =} \ConstantTok{TRUE}\NormalTok{)}
\end{Highlighting}
\end{Shaded}

\begin{verbatim}
[1] 68
\end{verbatim}

\subsection{Desvio padrão}\label{desvio-padruxe3o}

O desvio padrão de um dado objeto pode ser computado por meio da função
\emph{sd()}, conforme demonstrado a seguir:

\begin{Shaded}
\begin{Highlighting}[]
\FunctionTok{sd}\NormalTok{(df}\SpecialCharTok{$}\NormalTok{idade, }\AttributeTok{na.rm =} \ConstantTok{TRUE}\NormalTok{)}
\end{Highlighting}
\end{Shaded}

\begin{verbatim}
[1] 17.68238
\end{verbatim}

\subsection{Número de observações}\label{nuxfamero-de-observauxe7uxf5es}

O número de observações - ou comprimento - de um dado objeto pode ser
verificado com o uso da função \emph{length()}, conforme demonstrado a
seguir:

\begin{Shaded}
\begin{Highlighting}[]
\FunctionTok{length}\NormalTok{(df}\SpecialCharTok{$}\NormalTok{idade)}
\end{Highlighting}
\end{Shaded}

\begin{verbatim}
[1] 5
\end{verbatim}

Note que nesse caso é preciso despresar a omissão dos valores \emph{NA}.

\subsection{Exemplo: Criando uma tabela de estatísticas
descritivas}\label{exemplo-criando-uma-tabela-de-estatuxedsticas-descritivas}

Uma tabela de estatísticas descritivas mostra o perfil básico de um
banco de dados e geralmente expressa o número de observações, o valor
médio, o desvio padrão, o valor máximo e o valor mínimo de cada
variável. Para representar, considere usar o banco de dados nativo do
\emph{R} sobre características das flores (\emph{iris).} Esse banco de
dados possui cinco colunas (\emph{Sepal.Length, Sepal.Width,
Petal.Length, Petal.Width, Species}) que mostram o comprimento e a
largura da pétala e da sépala de cada espécie de flor. Elaborar uma
tabela de estatísticas descritivas dessa base de dados seria o mesmo que
preencher a seguinte tabela:

\begin{longtable}[]{@{}
  >{\raggedright\arraybackslash}p{(\columnwidth - 10\tabcolsep) * \real{0.2388}}
  >{\raggedright\arraybackslash}p{(\columnwidth - 10\tabcolsep) * \real{0.1940}}
  >{\raggedright\arraybackslash}p{(\columnwidth - 10\tabcolsep) * \real{0.1045}}
  >{\raggedright\arraybackslash}p{(\columnwidth - 10\tabcolsep) * \real{0.2239}}
  >{\raggedright\arraybackslash}p{(\columnwidth - 10\tabcolsep) * \real{0.1194}}
  >{\raggedright\arraybackslash}p{(\columnwidth - 10\tabcolsep) * \real{0.1194}}@{}}
\toprule\noalign{}
\begin{minipage}[b]{\linewidth}\raggedright
\end{minipage} & \begin{minipage}[b]{\linewidth}\raggedright
Observações
\end{minipage} & \begin{minipage}[b]{\linewidth}\raggedright
Média
\end{minipage} & \begin{minipage}[b]{\linewidth}\raggedright
Desvio Padrão
\end{minipage} & \begin{minipage}[b]{\linewidth}\raggedright
Mínimo
\end{minipage} & \begin{minipage}[b]{\linewidth}\raggedright
Máximo
\end{minipage} \\
\midrule\noalign{}
\endhead
\bottomrule\noalign{}
\endlastfoot
\emph{Sepal.Length} & & & & & \\
\emph{Sepal.Width} & & & & & \\
\emph{Petal.Length} & & & & & \\
\emph{Petal.Width} & & & & & \\
\end{longtable}

Para esse propósito, vamos usar a indexação aos nomes das linhas e
colunas e vamos usar as estatísticas básicas mostradas anteriormente. O
primeiro passo para tal é criar um data frame vazio com os mesmos
padrões da tabela anterior:

\begin{Shaded}
\begin{Highlighting}[]
\NormalTok{est\_desc }\OtherTok{=} \FunctionTok{data.frame}\NormalTok{(}
  \AttributeTok{Observacoes =} \FunctionTok{c}\NormalTok{(}\FunctionTok{rep}\NormalTok{(}\ConstantTok{NA}\NormalTok{, }\DecValTok{4}\NormalTok{)),}
  \AttributeTok{Media =} \FunctionTok{c}\NormalTok{(}\FunctionTok{rep}\NormalTok{(}\ConstantTok{NA}\NormalTok{, }\DecValTok{4}\NormalTok{)),}
  \AttributeTok{Desvio\_Padrao =} \FunctionTok{c}\NormalTok{(}\FunctionTok{rep}\NormalTok{(}\ConstantTok{NA}\NormalTok{, }\DecValTok{4}\NormalTok{)),}
  \AttributeTok{Minimo =} \FunctionTok{c}\NormalTok{(}\FunctionTok{rep}\NormalTok{(}\ConstantTok{NA}\NormalTok{, }\DecValTok{4}\NormalTok{)),}
  \AttributeTok{Maximo  =} \FunctionTok{c}\NormalTok{(}\FunctionTok{rep}\NormalTok{(}\ConstantTok{NA}\NormalTok{, }\DecValTok{4}\NormalTok{))}
\NormalTok{)}
\FunctionTok{rownames}\NormalTok{(est\_desc) }\OtherTok{=} \FunctionTok{c}\NormalTok{(}\StringTok{"Sepal.Length"}\NormalTok{, }\StringTok{"Sepal.Width"}\NormalTok{, }\StringTok{"Petal.Length"}\NormalTok{, }\StringTok{\textquotesingle{}Petal.Width\textquotesingle{}}\NormalTok{)}
\FunctionTok{print}\NormalTok{(est\_desc)}
\end{Highlighting}
\end{Shaded}

\begin{verbatim}
             Observacoes Media Desvio_Padrao Minimo Maximo
Sepal.Length          NA    NA            NA     NA     NA
Sepal.Width           NA    NA            NA     NA     NA
Petal.Length          NA    NA            NA     NA     NA
Petal.Width           NA    NA            NA     NA     NA
\end{verbatim}

O próximo passo é preencher cada célula do data frame com as
estatísticas correspondentes. Inicialmente, vamos preencher a coluna
referente ao número de observações:

\begin{Shaded}
\begin{Highlighting}[]
\NormalTok{est\_desc[}\StringTok{"Sepal.Length"}\NormalTok{, }\StringTok{"Observacoes"}\NormalTok{] }\OtherTok{=} \FunctionTok{length}\NormalTok{(iris}\SpecialCharTok{$}\NormalTok{Sepal.Length)}
\NormalTok{est\_desc[}\StringTok{"Sepal.Width"}\NormalTok{, }\StringTok{"Observacoes"}\NormalTok{] }\OtherTok{=} \FunctionTok{length}\NormalTok{(iris}\SpecialCharTok{$}\NormalTok{Sepal.Width)}
\NormalTok{est\_desc[}\StringTok{"Petal.Length"}\NormalTok{, }\StringTok{"Observacoes"}\NormalTok{] }\OtherTok{=} \FunctionTok{length}\NormalTok{(iris}\SpecialCharTok{$}\NormalTok{Petal.Length)}
\NormalTok{est\_desc[}\StringTok{"Petal.Width"}\NormalTok{, }\StringTok{"Observacoes"}\NormalTok{] }\OtherTok{=} \FunctionTok{length}\NormalTok{(iris}\SpecialCharTok{$}\NormalTok{Petal.Width)}
\end{Highlighting}
\end{Shaded}

Agora vamos preencher a coluna da média:

\begin{Shaded}
\begin{Highlighting}[]
\NormalTok{est\_desc[}\StringTok{"Sepal.Length"}\NormalTok{, }\StringTok{"Media"}\NormalTok{] }\OtherTok{=} \FunctionTok{mean}\NormalTok{(iris}\SpecialCharTok{$}\NormalTok{Sepal.Length)}
\NormalTok{est\_desc[}\StringTok{"Sepal.Width"}\NormalTok{, }\StringTok{"Media"}\NormalTok{] }\OtherTok{=} \FunctionTok{mean}\NormalTok{(iris}\SpecialCharTok{$}\NormalTok{Sepal.Width)}
\NormalTok{est\_desc[}\StringTok{"Petal.Length"}\NormalTok{, }\StringTok{"Media"}\NormalTok{] }\OtherTok{=} \FunctionTok{mean}\NormalTok{(iris}\SpecialCharTok{$}\NormalTok{Petal.Length)}
\NormalTok{est\_desc[}\StringTok{"Petal.Width"}\NormalTok{, }\StringTok{"Media"}\NormalTok{] }\OtherTok{=} \FunctionTok{mean}\NormalTok{(iris}\SpecialCharTok{$}\NormalTok{Petal.Width)}
\end{Highlighting}
\end{Shaded}

Agora vamos preencher a coluna do desvio padrão:

\begin{Shaded}
\begin{Highlighting}[]
\NormalTok{est\_desc[}\StringTok{"Sepal.Length"}\NormalTok{, }\StringTok{"Desvio\_Padrao"}\NormalTok{] }\OtherTok{=} \FunctionTok{sd}\NormalTok{(iris}\SpecialCharTok{$}\NormalTok{Sepal.Length)}
\NormalTok{est\_desc[}\StringTok{"Sepal.Width"}\NormalTok{, }\StringTok{"Desvio\_Padrao"}\NormalTok{] }\OtherTok{=} \FunctionTok{sd}\NormalTok{(iris}\SpecialCharTok{$}\NormalTok{Sepal.Width)}
\NormalTok{est\_desc[}\StringTok{"Petal.Length"}\NormalTok{, }\StringTok{"Desvio\_Padrao"}\NormalTok{] }\OtherTok{=} \FunctionTok{sd}\NormalTok{(iris}\SpecialCharTok{$}\NormalTok{Petal.Length)}
\NormalTok{est\_desc[}\StringTok{"Petal.Width"}\NormalTok{, }\StringTok{"Desvio\_Padrao"}\NormalTok{] }\OtherTok{=} \FunctionTok{sd}\NormalTok{(iris}\SpecialCharTok{$}\NormalTok{Petal.Width)}
\end{Highlighting}
\end{Shaded}

Agora vamos fazer o mesmo para a coluna do valor mínimo:

\begin{Shaded}
\begin{Highlighting}[]
\NormalTok{est\_desc[}\StringTok{"Sepal.Length"}\NormalTok{, }\StringTok{"Minimo"}\NormalTok{] }\OtherTok{=} \FunctionTok{min}\NormalTok{(iris}\SpecialCharTok{$}\NormalTok{Sepal.Length)}
\NormalTok{est\_desc[}\StringTok{"Sepal.Width"}\NormalTok{, }\StringTok{"Minimo"}\NormalTok{] }\OtherTok{=} \FunctionTok{min}\NormalTok{(iris}\SpecialCharTok{$}\NormalTok{Sepal.Width)}
\NormalTok{est\_desc[}\StringTok{"Petal.Length"}\NormalTok{, }\StringTok{"Minimo"}\NormalTok{] }\OtherTok{=} \FunctionTok{min}\NormalTok{(iris}\SpecialCharTok{$}\NormalTok{Petal.Length)}
\NormalTok{est\_desc[}\StringTok{"Petal.Width"}\NormalTok{, }\StringTok{"Minimo"}\NormalTok{] }\OtherTok{=} \FunctionTok{min}\NormalTok{(iris}\SpecialCharTok{$}\NormalTok{Petal.Width)}
\end{Highlighting}
\end{Shaded}

Por fim, vamos preencher a coluna do valor máximo:

\begin{Shaded}
\begin{Highlighting}[]
\NormalTok{est\_desc[}\StringTok{"Sepal.Length"}\NormalTok{, }\StringTok{"Maximo"}\NormalTok{] }\OtherTok{=} \FunctionTok{max}\NormalTok{(iris}\SpecialCharTok{$}\NormalTok{Sepal.Length)}
\NormalTok{est\_desc[}\StringTok{"Sepal.Width"}\NormalTok{, }\StringTok{"Maximo"}\NormalTok{] }\OtherTok{=} \FunctionTok{max}\NormalTok{(iris}\SpecialCharTok{$}\NormalTok{Sepal.Width)}
\NormalTok{est\_desc[}\StringTok{"Petal.Length"}\NormalTok{, }\StringTok{"Maximo"}\NormalTok{] }\OtherTok{=} \FunctionTok{max}\NormalTok{(iris}\SpecialCharTok{$}\NormalTok{Petal.Length)}
\NormalTok{est\_desc[}\StringTok{"Petal.Width"}\NormalTok{, }\StringTok{"Maximo"}\NormalTok{] }\OtherTok{=} \FunctionTok{max}\NormalTok{(iris}\SpecialCharTok{$}\NormalTok{Petal.Width)}
\end{Highlighting}
\end{Shaded}

O resultado desse procedimento é o data frame a seguir:

\begin{Shaded}
\begin{Highlighting}[]
\FunctionTok{print}\NormalTok{(est\_desc, }\AttributeTok{digits =} \DecValTok{4}\NormalTok{)}
\end{Highlighting}
\end{Shaded}

\begin{verbatim}
             Observacoes Media Desvio_Padrao Minimo Maximo
Sepal.Length         150 5.843        0.8281    4.3    7.9
Sepal.Width          150 3.057        0.4359    2.0    4.4
Petal.Length         150 3.758        1.7653    1.0    6.9
Petal.Width          150 1.199        0.7622    0.1    2.5
\end{verbatim}

Existem maneiras mais rápidas e mais eficientes de preparar uma tabela
de estatísticas descritivas, porém, isso exige artifícios que só serão
vistos em capítulos posteriores.

\section{Exercício}\label{exercuxedcio}

Considere usar a base de dados nativa sobre carros \emph{mtcars}.

\begin{Shaded}
\begin{Highlighting}[]
\NormalTok{df }\OtherTok{=}\NormalTok{ mtcars}
\end{Highlighting}
\end{Shaded}

(1) Remova o carro Fiat X1-9.

(2) Usando o nome das linhas e das colunas, encontre o consumo
\emph{(mpg)} do Toyota Corolla.

(3) Crie uma nova coluna de nome \emph{difCorolla} mostrando a diferença
entre o consumo médio de cada carro e o consumo médio do Toyota Corolla.

(4) Crie uma nova coluna de nome \emph{consumo\_peso} mostrando o
consumo \emph{(mpg)} dos carros para cada tonelada de peso \emph{(wt)}.

(5) Apague os carros de câmbio manual (carros com \emph{am = 0}).

\bookmarksetup{startatroot}

\chapter{Classes de elementos}\label{classes-de-elementos}

Os dados podem assumir diversos formatos, por exemplo, números, textos,
imagens, etc. A depender da ocasião, o usuário precisará lidar com um
tipo de dado específico. Por exemplo, um economista prevendo preços no
mercado de ações utilizará objetos numéricos para representar o preço
dos ativos. Um técnico em radiologia coleta informações em imagem de um
paciente. Um especialista em análise de sentimento em redes sociais
coleta os textos das postagens para extrair delas informações úteis.
Cada tipo de informação tem uma utilidade e um propósito e cada tipo de
informação possui propriedades que o analista de dados precisa conhecer.

Um objeto criado possui múltiplos elementos e esses elementos podem ser
de diferentes tipos. Esses \emph{``tipos''} de elementos recebem o nome
de \emph{classe}. Existem operações que são aplicadas adequadamente a
uma classe de elementos específica, por exemplo, é possível somar dois
elementos numéricos mas é estranho pensar em somar duas palavras. Nesse
capítulo vamos conhecer as principais classes de dados, suas
características e as principais operações que podem ser realizadas com
cada classe.

\section{\texorpdfstring{Elementos textuais:
(\emph{strings})}{Elementos textuais: (strings)}}\label{elementos-textuais-strings}

Os elementos textuais sempre devem ser declarados entre aspas, sejam
aspas simples ou duplas, do contrário, a linguagem \emph{R} não
reconhecerá esse elemento como texto. A linguagem \emph{R} não suporta
operações matemáticas com elementos textuais, ao contrário de outras
linguagens como \emph{Python} onde operações matemáticas básicas podem
ser aplicadas a esses elementos.

Imagine que seja necessário criar um data frame com o nome dos estados e
os nomes das capitais da região Sul do Brasil. Nesse caso, os elemento
são textuais e cada nome de estado e capital deve ser informado entre
aspas conforme demonstrado a seguir:

\begin{Shaded}
\begin{Highlighting}[]
\NormalTok{sul }\OtherTok{=} \FunctionTok{data.frame}\NormalTok{(}
  \AttributeTok{estados =} \FunctionTok{c}\NormalTok{(}\StringTok{"Paraná"}\NormalTok{, }\StringTok{"Santa Cararina"}\NormalTok{, }\StringTok{"Rio Grande do Sul"}\NormalTok{),}
  \AttributeTok{capital =} \FunctionTok{c}\NormalTok{(}\StringTok{"Curitiba"}\NormalTok{, }\StringTok{"Florianópolis"}\NormalTok{, }\StringTok{"Porto Alegre"}\NormalTok{)}
\NormalTok{)}
\FunctionTok{print}\NormalTok{(}\StringTok{"Sul"}\NormalTok{)}
\end{Highlighting}
\end{Shaded}

\begin{verbatim}
[1] "Sul"
\end{verbatim}

Caso os elementos não sejam declarados entre aspas, a execução do
comando retornará um erro.

É possível checar qual a classe de um dado elemento ou de um dado objeto
por meio do comando \emph{class()}, informando o nome do objeto entre
parêntesis. Por exemplo, é possível checar se o objeto criado
anteriormente com o nome \emph{sul} é um data frame, um vetor, um array,
etc.

\begin{Shaded}
\begin{Highlighting}[]
\FunctionTok{class}\NormalTok{(sul)}
\end{Highlighting}
\end{Shaded}

\begin{verbatim}
[1] "data.frame"
\end{verbatim}

O comando \emph{class()} também pode ser usado para verificar se um
objeto possui ou não uma classe textual. Caso se trate de um objeto
textual, então o comando \emph{class()} retornará o output
\emph{``character'',} indicando que se trata de um caracter. Por
exemplo, vamos verificar a classe da coluna de nome \emph{estados} do
objeto de nome \emph{sul}:

\begin{Shaded}
\begin{Highlighting}[]
\FunctionTok{class}\NormalTok{(sul}\SpecialCharTok{$}\NormalTok{estados)}
\end{Highlighting}
\end{Shaded}

\begin{verbatim}
[1] "character"
\end{verbatim}

Elementos númericos ou outros tipos de elementos muitas vezes podem ser
reconhecidos, declarados ou transformados em caracteres. Nesse caso,
cabe ao usuário saber identificar a classe dos elementos e transformá-lo
na classe desejada. Para exemplificar, considere incluir no data frame
anterior o código de unidade federativa do ibge:

\begin{Shaded}
\begin{Highlighting}[]
\NormalTok{sul}\SpecialCharTok{$}\NormalTok{codigo\_uf }\OtherTok{=} \FunctionTok{c}\NormalTok{(}\DecValTok{41}\NormalTok{, }\DecValTok{42}\NormalTok{, }\DecValTok{43}\NormalTok{)}
\FunctionTok{print}\NormalTok{(sul)}
\end{Highlighting}
\end{Shaded}

\begin{verbatim}
            estados       capital codigo_uf
1            Paraná      Curitiba        41
2    Santa Cararina Florianópolis        42
3 Rio Grande do Sul  Porto Alegre        43
\end{verbatim}

Note que os números não estão em aspas, indicando que não são
caracteres. Abrindo o data frame, o usuário verá facilmente que as duas
primeiras colunas estão com os valores alinhados à esquerda, enquanto a
última coluna está alinhada à direita. Nesse caso, apenas visualizando
esse alinhamento é possível afirmar que as duas primeiras colunas têm
elementos textuais. No entanto, o usuário pode checar se as colunas são
de fato \emph{strings} usando o comando \emph{is.character()}, indicando
o nome do objeto ou elemento no parênteses. Por exemplo, para checar se
o código do estado é um elemento textual, proceda conforme a seguir:

\begin{Shaded}
\begin{Highlighting}[]
\FunctionTok{is.character}\NormalTok{(sul}\SpecialCharTok{$}\NormalTok{codigo\_uf)}
\end{Highlighting}
\end{Shaded}

\begin{verbatim}
[1] FALSE
\end{verbatim}

Note que o output foi \emph{FALSE}, indicando que não se trata de um
elemento textual. Caso o usuário precise dessa coluna como um elemento
de texto, então ele pode usar o comando \emph{as.character()}, indicando
o nome do objeto ou elemento no parênteses. Nesse caso, o objeto ou
elemento indicado será transformado forçadamente em um elemento textual.
Para exemplificar, vamos transformar o código do estado em um caracter:

\begin{Shaded}
\begin{Highlighting}[]
\NormalTok{sul}\SpecialCharTok{$}\NormalTok{codigo\_uf }\OtherTok{=} \FunctionTok{as.character}\NormalTok{(sul}\SpecialCharTok{$}\NormalTok{codigo\_uf)}
\end{Highlighting}
\end{Shaded}

Checando agora a classe da coluna, nota-se que se trata de um elemento
textual:

\begin{Shaded}
\begin{Highlighting}[]
\FunctionTok{class}\NormalTok{(sul}\SpecialCharTok{$}\NormalTok{codigo\_uf)}
\end{Highlighting}
\end{Shaded}

\begin{verbatim}
[1] "character"
\end{verbatim}

Existem operações no âmbito da análise de dados que são usadas com esse
tipo de elemento. Por exemplo, as análises de sentimento baseiam-se
fundamentalmente em coletas de dados de texto. Um exemplo factível é a
mensuração da sensibilidade do mercado financeiro em relação a
intensidade do texto da ata do banco central. Esse procedimento é feito
pela contagem de palavras pré-definidas na ata. Para esse tipo de
operação existem bibliotecas que auxiliam o usuário e que serão vistas
posteriormente quando estivermos abordando operações com \emph{strings}.

\section{Elementos numéricos}\label{elementos-numuxe9ricos}

Os elementos numéricos são todos os elementos representados na forma de
números, sejam eles de quaisquer um dos conjuntos numéricos. Por
exemplo, 2 é um elemento numérico do conjunto dos naturais inteiros, ao
passo que 2.5 é um elemento numérico do conjunto dos números racionais.
É possível checar se um elemento é numérico usando o comando
\emph{class()}, no entanto, é mais adequado usar o \emph{is.numeric()},
indicando o nome do elemento em parênteses. Para exemplificar, vamos
checar se a coluna \emph{codigo\_uf} do data frame de nome \emph{sul} é
numérico.

\begin{Shaded}
\begin{Highlighting}[]
\FunctionTok{is.numeric}\NormalTok{(sul}\SpecialCharTok{$}\NormalTok{codigo\_uf)}
\end{Highlighting}
\end{Shaded}

\begin{verbatim}
[1] FALSE
\end{verbatim}

Note que o resultado é \emph{FALSE}, indicando que não se trata de um
elemento numérico. Caso o usuário precise transformar essa coluna em um
elemento numérico, então ele pode usar o comando \emph{as.numeric()},
indicando o nome do elemento em parêntesis:

\begin{Shaded}
\begin{Highlighting}[]
\NormalTok{sul}\SpecialCharTok{$}\NormalTok{codigo\_uf }\OtherTok{=} \FunctionTok{as.numeric}\NormalTok{(sul}\SpecialCharTok{$}\NormalTok{codigo\_uf)}
\end{Highlighting}
\end{Shaded}

Agora podemos verificar novamente se a coluna é ou não numérica.

\begin{Shaded}
\begin{Highlighting}[]
\FunctionTok{is.numeric}\NormalTok{(sul}\SpecialCharTok{$}\NormalTok{codigo\_uf)}
\end{Highlighting}
\end{Shaded}

\begin{verbatim}
[1] TRUE
\end{verbatim}

Em um data frame, as colunas numéricas sempre estarão alinhadas à
direita e o usuário pode checar a classe dessa coluna posicionando o
cursor sobre o seu nome no data frame.

Existe, contudo, um problema associado ao uso do comando
\emph{as.numeric()}, dado que a transformação de um elemento não
numérico para um elemento numérico só funciona perfeitamente caso o
objeto não numérico seja composto por números declarados como
caracteres, como é o caso da coluna \emph{codigo\_uf.} Do contrário, o
resultado pode ser composto por \emph{NAs}. Para exemplificar, vamos
tentar transformar o nome dos estados em um objeto numérico:

\begin{Shaded}
\begin{Highlighting}[]
\FunctionTok{as.numeric}\NormalTok{(sul}\SpecialCharTok{$}\NormalTok{estados)}
\end{Highlighting}
\end{Shaded}

\begin{verbatim}
Warning: NAs introduzidos por coerção
\end{verbatim}

\begin{verbatim}
[1] NA NA NA
\end{verbatim}

Por esse motivo, antes de efetuar qualquer operação com um elemento
numérico, é fundamental checar a sua classe para garantir que o
resultado esteja de acordo com o esperado. Por exemplo, imagine o caso
em que seja preciso calcular o pib per capita de um estado, sendo que o
pib seja numérico e a população seja um caracter. Nesse caso, se o
usuário não conhece a classe da população e não a transforma em um
elemento numérico, o cálculo do pib per capita será feito dividindo um
número por uma palavra, o que é algo inviável, resultando em um elemento
do tipo \emph{NA}.

\subsection{\texorpdfstring{Subclasses dos numéricos: inteiros
\emph{(integer)}}{Subclasses dos numéricos: inteiros (integer)}}\label{subclasses-dos-numuxe9ricos-inteiros-integer}

Como mencionado anteriormente, os elementos numéricos podem estar
contidos em quaisquer um dos conjuntos numéricos. Uma subclasse bastante
comum na economia são é a subclasse dos inteiros. Essa subclasse abriga
os números inteiros positivos ou negativos e se caracteriza por abrigar
um \emph{L} após o número. Por exemplo, 10 é um elemento da classe
numérica, mas 10L é um elemento da subclasse dos inteiros pertencente à
classe dos numéricos.

Para checar se um elemento é inteiro, use o comando \emph{is.integer()},
indicando o elemento no parêntesis. Por exemplo, vamos checar se 10 é
inteiro:

\begin{Shaded}
\begin{Highlighting}[]
\FunctionTok{is.integer}\NormalTok{(}\DecValTok{10}\NormalTok{)}
\end{Highlighting}
\end{Shaded}

\begin{verbatim}
[1] FALSE
\end{verbatim}

Note que o resultado é \emph{FALSE}, indicando que 10 não é inteiro.
Isso ocore porque os números inteiros devem obrigatoriamente ser
sucedidos da letra \emph{L}. Informando corretamente tem-se:

\begin{Shaded}
\begin{Highlighting}[]
\FunctionTok{is.integer}\NormalTok{(}\DecValTok{10}\NormalTok{L)}
\end{Highlighting}
\end{Shaded}

\begin{verbatim}
[1] TRUE
\end{verbatim}

Para transformar um elemento numérico na subclasse dos inteiros, basta
usar a função \emph{as.integer()}, informando o elemento no parênteses.
Por exemplo, vamos transformar o código do estado no data frame de nome
\emph{sul} em inteiro.

\begin{Shaded}
\begin{Highlighting}[]
\NormalTok{sul}\SpecialCharTok{$}\NormalTok{codigo\_uf }\OtherTok{=} \FunctionTok{as.integer}\NormalTok{(sul}\SpecialCharTok{$}\NormalTok{codigo\_uf)}
\FunctionTok{print}\NormalTok{(sul)}
\end{Highlighting}
\end{Shaded}

\begin{verbatim}
            estados       capital codigo_uf
1            Paraná      Curitiba        41
2    Santa Cararina Florianópolis        42
3 Rio Grande do Sul  Porto Alegre        43
\end{verbatim}

Note que visivelmente não há mudanças nas propriedades da coluna
modificada, porém agora quando consultarmos se essa coluna pertence ao
conjunto dos inteiros, o output é verdadeiro:

\begin{Shaded}
\begin{Highlighting}[]
\FunctionTok{is.integer}\NormalTok{(sul}\SpecialCharTok{$}\NormalTok{codigo\_uf)}
\end{Highlighting}
\end{Shaded}

\begin{verbatim}
[1] TRUE
\end{verbatim}

\subsection{\texorpdfstring{Subclasses dos numéricos: racionais
\emph{(doubles)}}{Subclasses dos numéricos: racionais (doubles)}}\label{subclasses-dos-numuxe9ricos-racionais-doubles}

Os números racionais são declarados como \emph{doubles} na linguagem
\emph{R}. Esse conjunto abriga também os inteiros e os naturais, isto é,
um inteiro sempre será um \emph{double,} assim como um natural sempre
será um \emph{double}. Para checar se um número pertence a essa
categoria, use a função \emph{is.double()}, indicando o elemento de
interesse no parênteses.

\begin{Shaded}
\begin{Highlighting}[]
\FunctionTok{is.double}\NormalTok{(}\FloatTok{10.5555}\NormalTok{)}
\end{Highlighting}
\end{Shaded}

\begin{verbatim}
[1] TRUE
\end{verbatim}

De maneira análoga, um elemento numérico declarado como texto pode ser
trannsformado em racional usando a função \emph{as.double()}, indicando
o elemento de interesse no parênteses.

\begin{Shaded}
\begin{Highlighting}[]
\FunctionTok{as.double}\NormalTok{(}\StringTok{"10.5"}\NormalTok{)}
\end{Highlighting}
\end{Shaded}

\begin{verbatim}
[1] 10.5
\end{verbatim}

Em alguns casos, os números racionais possuem múltiplas casas decimais e
é preciso reduzir essas casas decimais arredondando a última casa. Isso
pode ser facilmente resolvido usando a função \emph{round()}, que possui
o seguinte modo de uso: \emph{round(nome do objeto ou elemento, número
de casas decimais)}. Por exemeplo, imagine que queiramos reduzir o
número 5.56413 para apenas uma casa decimal. Nesse caso, devemos
proceder conforme a seguir:

\begin{Shaded}
\begin{Highlighting}[]
\FunctionTok{round}\NormalTok{(}\FloatTok{5.564132}\NormalTok{, }\DecValTok{1}\NormalTok{)}
\end{Highlighting}
\end{Shaded}

\begin{verbatim}
[1] 5.6
\end{verbatim}

\section{\texorpdfstring{Elementos lógicos: \emph{(TRUE e
FALSE)}}{Elementos lógicos: (TRUE e FALSE)}}\label{elementos-luxf3gicos-true-e-false}

Os elementos lógicos podem assumir dois valores, verdadeiro
\emph{(TRUE)} ou falso \emph{(FALSE)}. Sempre que o elemento lógico for
verdadeiro, o \emph{R} atribui valor 1 a este elemento, ao passo que
sempre que o elemento lógico for falso, o \emph{R} atribui valor 0 a
este elemento. Assim, é possível aplicar operações matemáticas aos
elementos pertencentes a essa classe. Isso advém da premissa de que as
linguagens de programação são baseados em sistemas binários de afirmação
e negação comentada no capítulo 1.

Essa classe de elementos é bastante utilizada na economia para
representar variáveis binárias onde a categoria de interesse recebe o
valor unitário. Por exemplo, imagine uma pesquisa com foco no
diferencial de salário por sexo. Nesse tipo de pesquisa é ideal saber se
o indivíduo é homem ou mulher. Se a categoria de interesse for o sexo
masculino, então os homens recebem valor \emph{TRUE} e as mulheres
recebem vaor \emph{FALSE}. Isso equivale a atribuir 1 para os homens e 0
para as mulheres.

Para verificar se um elemento pertence a essa categoria, use a função
\emph{is.logical()}, indicando o nome do elemento entre o parênteses.
Para exemplificar, vamos criar uma nova coluna no data frame de estados
da regiao sul com o nome \emph{parana} que identifica se o estado em
questão é ou não o estado do Paraná.

\begin{Shaded}
\begin{Highlighting}[]
\NormalTok{sul}\SpecialCharTok{$}\NormalTok{parana }\OtherTok{=} \FunctionTok{c}\NormalTok{(}\DecValTok{1}\NormalTok{,}\DecValTok{0}\NormalTok{,}\DecValTok{0}\NormalTok{)}
\FunctionTok{print}\NormalTok{(sul)}
\end{Highlighting}
\end{Shaded}

\begin{verbatim}
            estados       capital codigo_uf parana
1            Paraná      Curitiba        41      1
2    Santa Cararina Florianópolis        42      0
3 Rio Grande do Sul  Porto Alegre        43      0
\end{verbatim}

Agora vamos checar se essa coluna é um elemento lógico.

\begin{Shaded}
\begin{Highlighting}[]
\FunctionTok{is.logical}\NormalTok{(sul}\SpecialCharTok{$}\NormalTok{parana)}
\end{Highlighting}
\end{Shaded}

\begin{verbatim}
[1] FALSE
\end{verbatim}

Note que o output é \emph{FALSE}, indicando que não se trata de um
elemento lógico, o que é esperado dado que se trata de um elemento
numérico. Para transformar essa coluna em um elemento lógico, basta usar
a função \emph{as.logical()}, indicando o nome do elemento de interesse
no parênteses.

\begin{Shaded}
\begin{Highlighting}[]
\NormalTok{sul}\SpecialCharTok{$}\NormalTok{parana }\OtherTok{=} \FunctionTok{as.logical}\NormalTok{(sul}\SpecialCharTok{$}\NormalTok{parana)}
\FunctionTok{print}\NormalTok{(sul)}
\end{Highlighting}
\end{Shaded}

\begin{verbatim}
            estados       capital codigo_uf parana
1            Paraná      Curitiba        41   TRUE
2    Santa Cararina Florianópolis        42  FALSE
3 Rio Grande do Sul  Porto Alegre        43  FALSE
\end{verbatim}

Agora vamos checar novamente se essa coluna é um elemento lógico.

\begin{Shaded}
\begin{Highlighting}[]
\FunctionTok{is.logical}\NormalTok{(sul}\SpecialCharTok{$}\NormalTok{parana)}
\end{Highlighting}
\end{Shaded}

\begin{verbatim}
[1] TRUE
\end{verbatim}

Note que o output é \emph{TRUE}, indicando que se trata de um elemento
lógico.

\section{\texorpdfstring{Valores multicategóricos: Fatores
(\emph{factors})}{Valores multicategóricos: Fatores (factors)}}\label{valores-multicateguxf3ricos-fatores-factors}

Os \emph{factors} são elementos usados para representar valores
multicategóricos. Por exemplo, imagine uma variável que expressa a
situação do empregado no mercado de trabalho. Ele pode estar (1) apenas
trabalhando, (2) apenas estudando, (3) trabalhando e estudando, (4) nem
trabalhando nem estudando porém procurando emprego, ou (5) nem
trabalhando nem estudando nem procurando emprego. Note que são cinco
possibilidades que agora não podem ser representadas pelos elementos
lógicos.

Para declarar um elemento multicategórico é necessário usar o comando
\emph{factor()} que tem a seguinte forma de uso: \emph{factor(x =
elemento, levels = níveis das categorias)}. Para exemplificar, vamos
criar um objeto de nome \emph{emprego} com as possibilidades indicadas
no parágrafo anterior e os seus respectivos valores.

\begin{Shaded}
\begin{Highlighting}[]
\NormalTok{emprego }\OtherTok{=} \FunctionTok{factor}\NormalTok{(}
  \AttributeTok{x =} \FunctionTok{c}\NormalTok{(}
    \StringTok{"apenas trabalhando"}\NormalTok{, }
    \StringTok{"apenas estudando"}\NormalTok{, }
    \StringTok{"trabalhando e estudando"}\NormalTok{, }
    \StringTok{"nem trabalhando nem estudando porém procurando emprego"}\NormalTok{,}
    \StringTok{"nem trabalhando nem estudando nem procurando emprego"}
\NormalTok{  ),}
  \AttributeTok{levels =} \FunctionTok{c}\NormalTok{(}
    \StringTok{"apenas trabalhando"}\NormalTok{, }
    \StringTok{"apenas estudando"}\NormalTok{, }
    \StringTok{"trabalhando e estudando"}\NormalTok{, }
    \StringTok{"nem trabalhando nem estudando porém procurando emprego"}\NormalTok{,}
    \StringTok{"nem trabalhando nem estudando nem procurando emprego"}
\NormalTok{  )}
\NormalTok{)}

\FunctionTok{print}\NormalTok{(emprego)}
\end{Highlighting}
\end{Shaded}

\begin{verbatim}
[1] apenas trabalhando                                    
[2] apenas estudando                                      
[3] trabalhando e estudando                               
[4] nem trabalhando nem estudando porém procurando emprego
[5] nem trabalhando nem estudando nem procurando emprego  
5 Levels: apenas trabalhando apenas estudando ... nem trabalhando nem estudando nem procurando emprego
\end{verbatim}

Será atribuído valor 1 para a primeira categoria indicada no vetor de
níveis \emph{(levels)}, 2 para a segunda categoria e assim
sucessivamente.

Para checar se um elemento é multicategórico, devs-se usar a função
\emph{is.factor(),} indicando o nome do elemento no parênteses.

\begin{Shaded}
\begin{Highlighting}[]
\FunctionTok{is.factor}\NormalTok{(emprego)}
\end{Highlighting}
\end{Shaded}

\begin{verbatim}
[1] TRUE
\end{verbatim}

\section{Exercício 1}\label{exercuxedcio-1-1}

Considere o seguinte data frame:

\begin{Shaded}
\begin{Highlighting}[]
\FunctionTok{set.seed}\NormalTok{(}\DecValTok{10}\NormalTok{)}
\NormalTok{dados }\OtherTok{=} \FunctionTok{data.frame}\NormalTok{(}
  \AttributeTok{pessoa =} \DecValTok{1}\SpecialCharTok{:}\DecValTok{30}\NormalTok{,}
  \AttributeTok{idade =} \FunctionTok{sample}\NormalTok{(}\DecValTok{8}\SpecialCharTok{:}\DecValTok{85}\NormalTok{,}\DecValTok{30}\NormalTok{, }\AttributeTok{replace =}\NormalTok{ T),}
  \AttributeTok{sexo =} \FunctionTok{sample}\NormalTok{(}\FunctionTok{c}\NormalTok{(}\StringTok{"M"}\NormalTok{, }\StringTok{"F"}\NormalTok{), }\DecValTok{30}\NormalTok{, }\AttributeTok{replace =}\NormalTok{ T),}
  \AttributeTok{estado\_civil =} \FunctionTok{sample}\NormalTok{(}\FunctionTok{c}\NormalTok{(}\StringTok{"Solteiro"}\NormalTok{, }\StringTok{"Casado"}\NormalTok{, }\StringTok{"Viúvo"}\NormalTok{, }\StringTok{"Divorciado"}\NormalTok{), }\DecValTok{30}\NormalTok{, }\AttributeTok{replace =}\NormalTok{ T),}
  \AttributeTok{salario =} \FunctionTok{rnorm}\NormalTok{(}\DecValTok{30}\NormalTok{, }\AttributeTok{mean =} \DecValTok{1200}\NormalTok{, }\AttributeTok{sd =} \DecValTok{300}\NormalTok{)}
\NormalTok{)}
\end{Highlighting}
\end{Shaded}

(1) Crie uma variável de nome \emph{sexo2} transformando a variável
\emph{sexo} em um elemento lógico atribuindo o valor unitário para as
mulheres.

(2) Crie uma variável de nome \emph{fase\_vida} atribuindo os nomes
\emph{infância} para as pessoas com menos de 12 anos,
\emph{adolescência} para as pessoas com idade entre 12 e 18 anos,
\emph{adulta} para as pessoas com idade entre 18 e 65 anos e
\emph{velhice} para as pessoas com mais de 65 anos.

(3) Crie uma nova variável de nome \emph{fase\_vida2} transformando a
variável \emph{fase\_vida} em um factor ordenando as categorias de
acordo com a fase da vida em ordem crescente.

\section{Exercício 2}\label{exercuxedcio-2-1}

Considere a base de dados sobre carros \emph{mtcars}:

\begin{Shaded}
\begin{Highlighting}[]
\NormalTok{carros }\OtherTok{=}\NormalTok{ mtcars}
\end{Highlighting}
\end{Shaded}

(1) Crie uma coluna de nome \emph{automatico} transformando a coluna
\emph{am} em um elemento lógico.

(2) Transforme a coluna \emph{cyl} em um factor onde o atributo \emph{x}
recebe os valores \emph{``4 cilindros'', ``6 cilindore''} e \emph{``8
cilindros}'' ordenados na forma crescente.

(3) Crie uma nova coluna de nome \emph{carro} contendo o nome dos carros
indicados nos nomes das linhas do data frame.

(4) Verifique a casse da coluna criada na questão.

(5) Verifique se a coluna criada nas questões 1 e 2 são \emph{factors}.

(6) Verifique se a coluna \emph{mpg} é inteiro e caso não seja
transforme-a em inteiro.

\bookmarksetup{startatroot}

\chapter*{References}\label{references}
\addcontentsline{toc}{chapter}{References}

\markboth{References}{References}

\phantomsection\label{refs}
\begin{CSLReferences}{0}{1}
\end{CSLReferences}




\end{document}
